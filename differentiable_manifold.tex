\documentclass{report}

\input{preamble}


\begin{document}

% Your document content here
\chapter{Basic Concepts of Manifolds}
\section{Topological Manifold}
\subsection{Definition of Topological Manifold}
\dfn{Topological Manifold}{
    An $n$-dimensional \emph{topological manifold} is a Hausdorff topological space that is locally homeomorphic to $n$-dimensional Euclidean space.
}
``Locally homeomorphic to $n$-dimensional Euclidean space" means that for any point $p$ in the topological space $M$, there exists a neighborhood $N(p)$ of $p$ such that $N(p)$ is homeomorphic to the Euclidean space $\mathbb{R}^n$.

\subsection{Coordinate Chart}

\dfn{Coordinate Chart}{
    If $U_\alpha$ is an open set in $\mathbb{R}^n$,
    $$
        \mathbf{x}_\alpha:U_\alpha\longrightarrow \mathbf{x}_\alpha (U_\alpha)\subseteq M
    $$
    is a homeomorphism and $\mathbf{x}_\alpha (U_\alpha)$ contains some point $p\in M$, then the pair $(U_\alpha,\mathbf{x}_\alpha)$ is called a \emph{coordinate chart} or \emph{(local) coordinate system} of the topological manifold $M$ at point $p$.
}

Given a coordinate chart, $\mathbf{x}^{-1}_{\alpha}$ is often called the local coordinate mapping at point $p$. The image of point $p$ under $\mathbf{x}^{-1}_{\alpha}$, denoted as $\mathbf{x}^{-1}_\alpha(p)$, is called the local coordinate of point $p$. Coordinate charts can transform the local structure of a topological manifold into the familiar coordinates in $\mathbb{R}^n$, so they are sometimes also called parameterizations.

At any point $p$ in the topological manifold $M$, there exists at least one coordinate chart. In fact, let $\eta$ be a homeomorphism from $N(p)$ to $\mathbb{R}^n$. By definition of a neighborhood, there exists an open set $W\subseteq M$ such that $p\in W\subseteq N(p)$. Therefore, if we restrict $\eta$ to $W$ and let $\mathbf{x}=(\left.\eta\right|_{W})^{-1}$, then
$$
    \mathbf{x}:U\longrightarrow \mathbf{x}(U)
$$
is also a homeomorphism. Hence, $(U,\mathbf{x})$ is a coordinate chart at point $p$.

\subsection{Atlas}
\dfn{Atlas}{
If $\Sigma=\{(U_\alpha,\mathbf{x}_\alpha)\}_{\alpha\in A}$ is a collection of coordinate charts for the topological manifold $M$ and $\{\mathbf{x}_\alpha(U_\alpha)\}_{\alpha\in
    A}$ is a cover of $M$, then $\Sigma$ is called an \emph{atlas} of $M$.
}
A topological manifold $M$ can be regarded as a topological space obtained by gluing together many open subsets $U_\alpha(\alpha\in A)$ of Euclidean spaces.

Given an atlas $\Sigma=\{(U_\alpha,\mathbf{x}_\alpha)\}_{\alpha\in
    A}$ of the topological manifold $M$, some coordinate charts in the atlas may have overlapping regions. In other words, some points in $M$ may be covered by multiple coordinate charts. Suppose $(U_\alpha,\mathbf{x}_\alpha)$ and $(U_\beta,\mathbf{x}_\beta)$ have overlapping regions, i.e.,
\[
    \mathbf{x}_\alpha(U_\alpha)\cap \mathbf{x}_\beta(U_\beta)\ne\varnothing.
\]
Then we can define transition functions
\[
    \mathbf{x}^{-1}_\beta\circ \mathbf{x}_{\alpha}:
    \mathbf{x}^{-1}_\alpha(\mathbf{x}_\alpha(U_\alpha)\cap
    \mathbf{x}_\beta(U_\beta)) \longrightarrow
    \mathbf{x}^{-1}_\beta(\mathbf{x}_\alpha(U_\alpha)\cap
    \mathbf{x}_\beta(U_\beta))
\]
and
\[
    \mathbf{x}^{-1}_{\alpha}\circ
    \mathbf{x}_\beta:  \mathbf{x}^{-1}_\beta(\mathbf{x}_\alpha(U_\alpha)\cap
    \mathbf{x}_\beta(U_\beta))\longrightarrow
    \mathbf{x}^{-1}_\alpha(\mathbf{x}_\alpha(U_\alpha)\cap
    \mathbf{x}_\beta(U_\beta)).
\]
Since the composition of two homeomorphisms is still a homeomorphism, the above two transition functions are both homeomorphisms. As mentioned earlier, a point $p$ in $M$ may be covered by multiple coordinate charts, meaning that point $p$ may have multiple local coordinates. With the help of transition functions, we can switch between multiple local coordinates of point $p$.

\section{Differentiable Manifold}
\subsection{Motivation: Defining Differentiable Functions}
A topological manifold is a type of topological space that is locally ``very similar" to Euclidean space. Here, ``very similar" refers to a consistent topological structure. Since topological manifolds have a topology that locally behaves like the Euclidean topology, various techniques developed for the Euclidean topology can be applied to topological manifolds. 

As is well known, Euclidean spaces also have many structures more refined than topology, such as the ability to define differentiable functions and compute derivatives. If a topological manifold locally inherits this more refined structure, then, similarly, we can define differentiable functions and compute derivatives on the topological manifold. We will now provide a formal definition of this more refined structure.

Let $M$ be an $n$-dimensional topological manifold and $\Sigma=\{(U_\alpha,\mathbf{x}_\alpha)\}_{\alpha\in
A}$ be an atlas of $M$. Our goal is to define differentiable functions $f:M\to \mathbb{R}^m$ on a topological manifold $M$ in a reasonable way. A natural idea is to locally pull back the domain from a subset of $M$ to $U_\alpha\subset\mathbb{R}^n$ using the coordinate chart $(U_\alpha,\mathbf{x}_\alpha)$. Specifically, we can attempt to define differentiable functions as follows: function $f:M\to \mathbb{R}^m$ is differentiable at point $p$ if and only if for any coordinate chart $(U_\alpha,\mathbf{x}_\alpha)$ covering point $p$, the composite function
\[
    f\circ \mathbf{x}_\alpha:U_\alpha\longrightarrow \mathbb{R}^m
\]
is differentiable at point $\mathbf{x}^{-1}_\alpha(p)$.

\subsection{Compatible Atlas}
This definition is theoretically valid but can be cumbersome in practice. To prove the differentiability of $f$ at point $p$, we need to verify differentiablity of $f\circ \mathbf{x}_\alpha$ for all coordinate charts $(U_\alpha,\mathbf{x}_\alpha)$ covering point $p$. 

In fact, if we only consider so-called compatible atlases, the situation becomes much simpler. In this case, we only need to verify one arbatrary 
coordinate chart covering point $p$. The following definition makes this idea precise.

\dfn{Compatible Coordinate Charts and Compatible Atlas}{Two coordinate charts $(U_\alpha,\mathbf{x}_\alpha)$, $(U_\beta,\mathbf{x}_\beta)$ are called \emph{compatible} if
\[
    U_\alpha\cap U_\beta\ne\varnothing\implies\mathbf{x}^{-1}_\beta\circ
\mathbf{x}_{\alpha},\;\mathbf{x}^{-1}_{\alpha}\circ \mathbf{x}_\beta
\text{ are differentiable}.
\]
If any two coordinate charts in an atlas are compatible, the atlas is called a \emph{compatible atlas}. 
}

Suppose $(U_\alpha,\mathbf{x}_\alpha)$, $(U_\beta,\mathbf{x}_\beta)$ are two coordinate charts covering point $p$. Notice that
$$
\begin{aligned}
f\circ \mathbf{x}_\beta &= (f\circ \mathbf{x}_\alpha)\circ(\mathbf{x}^{-1}_\alpha\circ \mathbf{x}_{\beta}),\\
f\circ \mathbf{x}_\alpha &= (f\circ \mathbf{x}_\beta)\circ(\mathbf{x}^{-1}_\beta\circ \mathbf{x}_{\alpha}).
\end{aligned}
$$
We see that $f \circ \mathbf{x}_\beta$ is differentiable if and only if $f \circ \mathbf{x}_\alpha$ is differentiable. Therefore, if the given atlas is compatible, our definition of differentiable functions can be written as: a function $f: M \to \mathbb{R}^m$ is differentiable at point $p$ if and only if there exists a coordinate chart $(U_\alpha, \mathbf{x}_\alpha)$ covering point $p$ such that the composite function
$$
f\circ \mathbf{x}_\alpha:U_\alpha\longrightarrow \mathbb{R}^m
$$
is differentiable at point $\mathbf{x}^{-1}_\alpha(p)$.

\subsection{Differential Structure and Differentiable Manifold}
Compatible atlases can be maximized.
\dfn{Maximal Atlas of a Compatible Atlas}{
    Let $\Sigma=\{(U_\alpha,\mathbf{x}_\alpha)\}_{\alpha\in A}$ be a compatible atlas of the topological manifold $(M, \tau)$, then
\[
\Sigma_{\max}=\left\{(U_\alpha,\mathbf{x}_\alpha)\mid(U_\alpha,\mathbf{x}_\alpha) \text{ is compatible with every coordinate chart in $\Sigma$}\right\}
\]
is called the maximal atlas of $\Sigma$.
} 
A maximal compatible atlas $\Sigma_{\max}$ on the topological manifold $(M, \tau)$ is called a \emph{differential structure} of $(M, \tau)$. A topological manifold equipped with a differential structure is called a differentiable manifold, denoted as the triplet $(M, \tau, \Sigma_{\max})$. Unless otherwise specified, when we talk about coordinate charts of differentiable manifolds, we always assume that the coordinate charts are taken from the differentiable manifold's own atlas.

By replacing ``differentiable" with ``$C^r$-differentiable" or ``smooth", we can similarly define $C^r$-compatible atlases and smooth compatible atlases, $C^r$-differential structures and smooth structures, and $C^r$-differentiable manifolds and smooth manifolds. In the following, we will mainly discuss smooth manifolds, but the related concepts can be extended to differential manifolds and $C^r$-differential manifolds in a similar way.











% \begin{Theorem}{Title}{label}
%     This is the statement of the theorem.
% \end{Theorem}

% \begin{corollary}{Title}{label}
%     This is the statement of the corollary.
% \end{corollary}

% \begin{claim}{Title}{label}
%     This is the statement of the claim.
% \end{claim}

% \begin{Example}{Title}{label}
%     This is an example.
% \end{Example}

% \begin{Definition}{Title}{label}
%     This is a definition.
% \end{Definition}
\end{document}