\documentclass{report}

%%%%%%%%%%%%%%%%%%%%%%%%%%%%%%%%%
% PACKAGE IMPORTS
%%%%%%%%%%%%%%%%%%%%%%%%%%%%%%%%%


\usepackage[tmargin=2cm,rmargin=1in,lmargin=1in,margin=0.85in,bmargin=2cm,footskip=.2in]{geometry}
\usepackage{amsmath,amsfonts,amsthm,amssymb,mathtools}
\usepackage{bookmark}
\usepackage{enumitem}
\usepackage{hyperref,theoremref}
\hypersetup{
	pdftitle={Differentiable Manifolds},
	colorlinks=true, linkcolor=doc!90,
	bookmarksnumbered=true,
	bookmarksopen=true
}
\usepackage[most,many,breakable]{tcolorbox}
\usepackage{xcolor}
\usepackage{graphicx}
\graphicspath{ {./images/} }
\usepackage{varwidth}
\usepackage{varwidth}
\usepackage{etoolbox}
%\usepackage{authblk}
\usepackage{nameref}
\usepackage{multicol,array}
\usepackage{tikz-cd}
\usepackage{cancel}
\usepackage{pgfplots}
\pgfplotsset{compat=newest}
\usepgfplotslibrary{patchplots}
\usepackage{anyfontsize}
\usepackage{sectsty}
%\usepackage{import}
%\usepackage{xifthen}
%\usepackage{pdfpages}
%\usepackage{transparent}


%%%%%%%%%%%%%%%%%%%%%%%%%%%%%%
% SELF MADE COLORS
%%%%%%%%%%%%%%%%%%%%%%%%%%%%%%

\usetikzlibrary{ shapes.geometric }
\usetikzlibrary{calc}
\usepackage{anyfontsize}

\definecolor{myg}{RGB}{56, 140, 70}
\definecolor{myb}{RGB}{45, 111, 177}
\definecolor{myr}{RGB}{199, 68, 64}
\definecolor{mytheorembg}{HTML}{fdf8ea}
\definecolor{mytheoremfr}{HTML}{f19000}
\definecolor{myexamplebg}{HTML}{F2FBF8}
\definecolor{myexamplefr}{HTML}{88D6D1}
\definecolor{myexampleti}{HTML}{2A7F7F}
\definecolor{mydefinitbg}{HTML}{F2F2F9}
\definecolor{mydefinitfr}{HTML}{00007B}
\definecolor{notesgreen}{RGB}{0,162,0}
\definecolor{myp}{RGB}{197, 92, 212}
\definecolor{mygr}{HTML}{2C3338}
\definecolor{myred}{RGB}{127,0,0}
\definecolor{myyellow}{RGB}{169,121,69}
\definecolor{OrangeRed}{HTML}{ED135A}
\definecolor{Dandelion}{HTML}{FDBC42}
\definecolor{light-gray}{gray}{0.95}
\definecolor{Emerald}{HTML}{00A99D}
\definecolor{RoyalBlue}{HTML}{0071BC}

% \definecolor{mydefnewbg}{HTML}{FFF0E0}
% \definecolor{mydefnewfr}{HTML}{FF9900}

%%%%%%%%%%%%%%%%%%%%%%%%%%%%
% TCOLORBOX SETUPS
%%%%%%%%%%%%%%%%%%%%%%%%%%%%

\setlength{\parindent}{0.5cm}
%================================
% THEOREM BOX
%================================

\tcbuselibrary{theorems,skins,hooks}
\newtcbtheorem[number within=section]{Theorem}{Theorem}
{%
	enhanced,
	breakable,
	colback = mytheorembg,
	frame hidden,
	boxrule = 0sp,
	borderline west = {2pt}{0pt}{mytheoremfr},
	sharp corners,
	detach title,
	before upper = \tcbtitle\par\smallskip,
	coltitle = mytheoremfr,
	fonttitle = \bfseries\sffamily,
	description font = \mdseries,
	separator sign none,
	segmentation style={solid, mytheoremfr},
}
{th}

\tcbuselibrary{theorems,skins,hooks}
\newtcbtheorem[number within=chapter]{theorem}{Theorem}
{%
	enhanced,
	breakable,
	colback = mytheorembg,
	frame hidden,
	boxrule = 0sp,
	borderline west = {2pt}{0pt}{mytheoremfr},
	sharp corners,
	detach title,
	before upper = \tcbtitle\par\smallskip,
	coltitle = mytheoremfr,
	fonttitle = \bfseries\sffamily,
	description font = \mdseries,
	separator sign none,
	segmentation style={solid, mytheoremfr},
}
{th}


\tcbuselibrary{theorems,skins,hooks}
\newtcolorbox{Theoremcon}
{%
	enhanced
	,breakable
	,colback = mytheorembg
	,frame hidden
	,boxrule = 0sp
	,borderline west = {2pt}{0pt}{mytheoremfr}
	,sharp corners
	,description font = \mdseries
	,separator sign none
}


%================================
% Corollery
%================================
\tcbuselibrary{theorems,skins,hooks}
\newtcbtheorem[number within=section]{corolary}{Corollary}
{%
	enhanced
	,breakable
	,colback = myp!10
	,frame hidden
	,boxrule = 0sp
	,borderline west = {2pt}{0pt}{myp!85!black}
	,sharp corners
	,detach title
	,before upper = \tcbtitle\par\smallskip
	,coltitle = myp!85!black
	,fonttitle = \bfseries\sffamily
	,description font = \mdseries
	,separator sign none
	,segmentation style={solid, myp!85!black}
}
{th}
\tcbuselibrary{theorems,skins,hooks}
\newtcbtheorem[number within=chapter]{corollary}{Corollary}
{%
	enhanced
	,breakable
	,colback = myp!10
	,frame hidden
	,boxrule = 0sp
	,borderline west = {2pt}{0pt}{myp!85!black}
	,sharp corners
	,detach title
	,before upper = \tcbtitle\par\smallskip
	,coltitle = myp!85!black
	,fonttitle = \bfseries\sffamily
	,description font = \mdseries
	,separator sign none
	,segmentation style={solid, myp!85!black}
}
{th}

%================================
% CLAIM
%================================

\tcbuselibrary{theorems,skins,hooks}
\newtcbtheorem[number within=section]{claim}{Claim}
{%
	enhanced
	,breakable
	,colback = myg!10
	,frame hidden
	,boxrule = 0sp
	,borderline west = {2pt}{0pt}{myg}
	,sharp corners
	,detach title
	,before upper = \tcbtitle\par\smallskip
	,coltitle = myg!85!black
	,fonttitle = \bfseries\sffamily
	,description font = \mdseries
	,separator sign none
	,segmentation style={solid, myg!85!black}
}
{th}


\newtcbtheorem[number within=chapter]{Claim}{Claim}
{%
	enhanced
	,breakable
	,colback = myg!10
	,frame hidden
	,boxrule = 0sp
	,borderline west = {2pt}{0pt}{myg}
	,sharp corners
	,detach title
	,before upper = \tcbtitle\par\smallskip
	,coltitle = myg!85!black
	,fonttitle = \bfseries\sffamily
	,description font = \mdseries
	,separator sign none
	,segmentation style={solid, myg!85!black}
}
{th}

%================================
% EXAMPLE BOX
%================================

\newtcbtheorem[number within=section]{Example}{Example}
{%
	colback = myexamplebg
	,breakable
	,colframe = myexamplefr
	,coltitle = myexampleti
	,boxrule = 1pt
	,sharp corners
	,detach title
	,before upper=\tcbtitle\par\smallskip
	,fonttitle = \bfseries
	,description font = \mdseries
	,separator sign none
	,description delimiters parenthesis
}
{ex}

\newtcbtheorem[number within=chapter]{example}{Example}
{%
	colback = myexamplebg
	,breakable
	,colframe = myexamplefr
	,coltitle = myexampleti
	,boxrule = 1pt
	,sharp corners
	,detach title
	,before upper=\tcbtitle\par\smallskip
	,fonttitle = \bfseries
	,description font = \mdseries
	,separator sign none
	,description delimiters parenthesis
}
{ex}

%================================
% DEFINITION BOX
%================================

% \newtcbtheorem[number within=section]{Definition}{Definition}{enhanced,
% 	before skip=2mm,after skip=2mm, colback=red!5,colframe=red!80!black,boxrule=0.5mm,
% 	attach boxed title to top left={xshift=1cm,yshift*=1mm-\tcboxedtitleheight}, varwidth boxed title*=-3cm,
% 	boxed title style={frame code={
% 					\path[fill=tcbcolback]
% 					([yshift=-1mm,xshift=-1mm]frame.north west)
% 					arc[start angle=0,end angle=180,radius=1mm]
% 					([yshift=-1mm,xshift=1mm]frame.north east)
% 					arc[start angle=180,end angle=0,radius=1mm];
% 					\path[left color=tcbcolback!60!black,right color=tcbcolback!60!black,
% 						middle color=tcbcolback!80!black]
% 					([xshift=-2mm]frame.north west) -- ([xshift=2mm]frame.north east)
% 					[rounded corners=1mm]-- ([xshift=1mm,yshift=-1mm]frame.north east)
% 					-- (frame.south east) -- (frame.south west)
% 					-- ([xshift=-1mm,yshift=-1mm]frame.north west)
% 					[sharp corners]-- cycle;
% 				},interior engine=empty,
% 		},
% 	fonttitle=\bfseries,
% 	title={#2},#1}{def}
% \newtcbtheorem[number within=chapter]{definition}{Definition}{enhanced,
% 	before skip=2mm,after skip=2mm, colback=red!5,colframe=red!80!black,boxrule=0.5mm,
% 	attach boxed title to top left={xshift=1cm,yshift*=1mm-\tcboxedtitleheight}, varwidth boxed title*=-3cm,
% 	boxed title style={frame code={
% 					\path[fill=tcbcolback]
% 					([yshift=-1mm,xshift=-1mm]frame.north west)
% 					arc[start angle=0,end angle=180,radius=1mm]
% 					([yshift=-1mm,xshift=1mm]frame.north east)
% 					arc[start angle=180,end angle=0,radius=1mm];
% 					\path[left color=tcbcolback!60!black,right color=tcbcolback!60!black,
% 						middle color=tcbcolback!80!black]
% 					([xshift=-2mm]frame.north west) -- ([xshift=2mm]frame.north east)
% 					[rounded corners=1mm]-- ([xshift=1mm,yshift=-1mm]frame.north east)
% 					-- (frame.south east) -- (frame.south west)
% 					-- ([xshift=-1mm,yshift=-1mm]frame.north west)
% 					[sharp corners]-- cycle;
% 				},interior engine=empty,
% 		},
% 	fonttitle=\bfseries,
% 	title={#2},#1}{def}

%================================
% NEW DEFINITION BOX
%================================

% Define new colors


\newtcbtheorem[number within=section]{Definition}{Definition}
{%
    enhanced
    ,breakable
    ,colback = mydefinitbg
    ,frame hidden
    ,boxrule = 0sp
    ,borderline west = {2pt}{0pt}{mydefinitfr}
    ,sharp corners
    ,detach title
    ,before upper = \tcbtitle\par\smallskip
    ,coltitle = mydefinitfr!85!black
    ,fonttitle = \bfseries\sffamily
    ,description font = \mdseries
    ,separator sign none
    ,segmentation style={solid, mydefinitfr!85!black}
}
{th}

\newtcbtheorem[number within=chapter]{definition}{Definition}
{%
    enhanced
    ,breakable
    ,colback = mydefinitbg
    ,frame hidden
    ,boxrule = 0sp
    ,borderline west = {2pt}{0pt}{mydefinitfr}
    ,sharp corners
    ,detach title
    ,before upper = \tcbtitle\par\smallskip
    ,coltitle = mydefinitfr!85!black
    ,fonttitle = \bfseries\sffamily
    ,description font = \mdseries
    ,separator sign none
    ,segmentation style={solid, mydefinitfr!85!black}
}
{th}

%================================
% OPEN QUESTION BOX
%================================

\newtcbtheorem[number within=section]{open}{Open Question}{enhanced,
	before skip=2mm,after skip=2mm, colback=myp!5,colframe=myp!80!black,boxrule=0.5mm,
	attach boxed title to top left={xshift=1cm,yshift*=1mm-\tcboxedtitleheight}, varwidth boxed title*=-3cm,
	boxed title style={frame code={
			\path[fill=tcbcolback]
			([yshift=-1mm,xshift=-1mm]frame.north west)
			arc[start angle=0,end angle=180,radius=1mm]
			([yshift=-1mm,xshift=1mm]frame.north east)
			arc[start angle=180,end angle=0,radius=1mm];
			\path[left color=tcbcolback!60!black,right color=tcbcolback!60!black,
			middle color=tcbcolback!80!black]
			([xshift=-2mm]frame.north west) -- ([xshift=2mm]frame.north east)
			[rounded corners=1mm]-- ([xshift=1mm,yshift=-1mm]frame.north east)
			-- (frame.south east) -- (frame.south west)
			-- ([xshift=-1mm,yshift=-1mm]frame.north west)
			[sharp corners]-- cycle;
		},interior engine=empty,
	},
	fonttitle=\bfseries,
	title={#2},#1}{def}
\newtcbtheorem[number within=chapter]{Open}{Open Question}{enhanced,
	before skip=2mm,after skip=2mm, colback=myp!5,colframe=myp!80!black,boxrule=0.5mm,
	attach boxed title to top left={xshift=1cm,yshift*=1mm-\tcboxedtitleheight}, varwidth boxed title*=-3cm,
	boxed title style={frame code={
			\path[fill=tcbcolback]
			([yshift=-1mm,xshift=-1mm]frame.north west)
			arc[start angle=0,end angle=180,radius=1mm]
			([yshift=-1mm,xshift=1mm]frame.north east)
			arc[start angle=180,end angle=0,radius=1mm];
			\path[left color=tcbcolback!60!black,right color=tcbcolback!60!black,
			middle color=tcbcolback!80!black]
			([xshift=-2mm]frame.north west) -- ([xshift=2mm]frame.north east)
			[rounded corners=1mm]-- ([xshift=1mm,yshift=-1mm]frame.north east)
			-- (frame.south east) -- (frame.south west)
			-- ([xshift=-1mm,yshift=-1mm]frame.north west)
			[sharp corners]-- cycle;
		},interior engine=empty,
	},
	fonttitle=\bfseries,
	title={#2},#1}{def}



%================================
% EXERCISE BOX
%================================

\makeatletter
\newtcbtheorem{question}{Question}{enhanced,
	breakable,
	colback=white,
	colframe=myb!80!black,
	attach boxed title to top left={yshift*=-\tcboxedtitleheight},
	fonttitle=\bfseries,
	title={#2},
	boxed title size=title,
	boxed title style={%
			sharp corners,
			rounded corners=northwest,
			colback=tcbcolframe,
			boxrule=0pt,
		},
	underlay boxed title={%
			\path[fill=tcbcolframe] (title.south west)--(title.south east)
			to[out=0, in=180] ([xshift=5mm]title.east)--
			(title.center-|frame.east)
			[rounded corners=\kvtcb@arc] |-
			(frame.north) -| cycle;
		},
	#1
}{def}
\makeatother

%================================
% SOLUTION BOX
%================================

\makeatletter
\newtcolorbox{solution}{enhanced,
	breakable,
	colback=white,
	colframe=myg!80!black,
	attach boxed title to top left={yshift*=-\tcboxedtitleheight},
	title=Solution,
	boxed title size=title,
	boxed title style={%
			sharp corners,
			rounded corners=northwest,
			colback=tcbcolframe,
			boxrule=0pt,
		},
	underlay boxed title={%
			\path[fill=tcbcolframe] (title.south west)--(title.south east)
			to[out=0, in=180] ([xshift=5mm]title.east)--
			(title.center-|frame.east)
			[rounded corners=\kvtcb@arc] |-
			(frame.north) -| cycle;
		},
}
\makeatother

%================================
% Question BOX
%================================

\makeatletter
\newtcbtheorem{qstion}{Question}{enhanced,
	breakable,
	colback=white,
	colframe=mygr,
	attach boxed title to top left={yshift*=-\tcboxedtitleheight},
	fonttitle=\bfseries,
	title={#2},
	boxed title size=title,
	boxed title style={%
			sharp corners,
			rounded corners=northwest,
			colback=tcbcolframe,
			boxrule=0pt,
		},
	underlay boxed title={%
			\path[fill=tcbcolframe] (title.south west)--(title.south east)
			to[out=0, in=180] ([xshift=5mm]title.east)--
			(title.center-|frame.east)
			[rounded corners=\kvtcb@arc] |-
			(frame.north) -| cycle;
		},
	#1
}{def}
\makeatother

\newtcbtheorem[number within=chapter]{wconc}{Wrong Concept}{
	breakable,
	enhanced,
	colback=white,
	colframe=myr,
	arc=0pt,
	outer arc=0pt,
	fonttitle=\bfseries\sffamily\large,
	colbacktitle=myr,
	attach boxed title to top left={},
	boxed title style={
			enhanced,
			skin=enhancedfirst jigsaw,
			arc=3pt,
			bottom=0pt,
			interior style={fill=myr}
		},
	#1
}{def}



%================================
% NOTE BOX
%================================

\usetikzlibrary{arrows,calc,shadows.blur}
\tcbuselibrary{skins}
\newtcolorbox{note}[1][]{%
	enhanced jigsaw,
	colback=gray!20!white,%
	colframe=gray!80!black,
	size=small,
	boxrule=1pt,
	title=\textbf{Note:-},
	halign title=flush center,
	coltitle=black,
	breakable,
	drop shadow=black!50!white,
	attach boxed title to top left={xshift=1cm,yshift=-\tcboxedtitleheight/2,yshifttext=-\tcboxedtitleheight/2},
	minipage boxed title=1.5cm,
	boxed title style={%
			colback=white,
			size=fbox,
			boxrule=1pt,
			boxsep=2pt,
			underlay={%
					\coordinate (dotA) at ($(interior.west) + (-0.5pt,0)$);
					\coordinate (dotB) at ($(interior.east) + (0.5pt,0)$);
					\begin{scope}
						\clip (interior.north west) rectangle ([xshift=3ex]interior.east);
						\filldraw [white, blur shadow={shadow opacity=60, shadow yshift=-.75ex}, rounded corners=2pt] (interior.north west) rectangle (interior.south east);
					\end{scope}
					\begin{scope}[gray!80!black]
						\fill (dotA) circle (2pt);
						\fill (dotB) circle (2pt);
					\end{scope}
				},
		},
	#1,
}

%%%%%%%%%%%%%%%%%%%%%%%%%%%%%%
% SELF MADE COMMANDS
%%%%%%%%%%%%%%%%%%%%%%%%%%%%%%

\NewDocumentCommand{\EqM}{ m O{black} m}{%
	\tikz[remember picture, baseline, anchor=base] 
	\node[inner sep=0pt, outer sep=3pt, text=#2] (#1) {%
		\ensuremath{#3}%
	};    
}

\newcommand{\thm}[3][]{\begin{Theorem}{#2}{#1}#3\end{Theorem}}
\newcommand{\thmc}[3][]{\begin{theorem}{#2}{#1}#3\end{theorem}}
\newcommand{\cor}[3][]{\begin{corolary}{#2}{#1}#3\end{corolary}}
\newcommand{\corc}[3][]{\begin{corollary}{#2}{#1}#3\end{corollary}}
\newcommand{\clm}[3][]{\begin{claim}{#2}{#1}#3\end{claim}}
\newcommand{\wc}[3][]{\begin{wconc}{#2}{#1}\setlength{\parindent}{1cm}#3\end{wconc}}
\newcommand{\thmcon}[1]{\begin{Theoremcon}{#1}\end{Theoremcon}}
\newcommand{\ex}[3][]{\begin{Example}{#2}{#1}#3\end{Example}}
\newcommand{\exc}[3][]{\begin{example}{#2}{#1}#3\end{example}}
\newcommand{\dfn}[3][]{\begin{Definition}[colbacktitle=red!75!black]{#2}{#1}#3\end{Definition}}
\newcommand{\dfnc}[3][]{\begin{definition}[colbacktitle=red!75!black]{#2}{#1}#3\end{definition}}
\newcommand{\opn}[3][]{\begin{open}[colbacktitle=myp!75!black]{#2}{#1}#3\end{open}}
\newcommand{\opnc}[3][]{\begin{Open}[colbacktitle=myp!75!black]{#2}{#1}#3\end{Open}}
\newcommand{\qs}[3][]{\begin{question}{#2}{#1}#3\end{question}}
\newcommand{\pf}[2]{\begin{myproof}[#1]#2\end{myproof}}
\newcommand{\nt}[1]{\begin{note}#1\end{note}}

\newcommand*\circled[1]{\tikz[baseline=(char.base)]{
		\node[shape=circle,draw,inner sep=1pt] (char) {#1};}}
\newcommand\getcurrentref[1]{%
	\ifnumequal{\value{#1}}{0}
	{??}
	{\the\value{#1}}%
}
\newcommand{\getCurrentSectionNumber}{\getcurrentref{section}}
\newenvironment{myproof}[1][\proofname]{%
	\proof[\bfseries #1: ]%
}{\endproof}
\newcounter{mylabelcounter}

\makeatletter
\newcommand{\setword}[2]{%
	\phantomsection
	#1\def\@currentlabel{\unexpanded{#1}}\label{#2}%
}
\makeatother




\tikzset{
	symbol/.style={
			draw=none,
			every to/.append style={
					edge node={node [sloped, allow upside down, auto=false]{$#1$}}}
		}
}

%\usepackage{framed}
%\usepackage{titletoc}
%\usepackage{etoolbox}
%\usepackage{lmodern}


%\patchcmd{\tableofcontents}{\contentsname}{\sffamily\contentsname}{}{}

%\renewenvironment{leftbar}
%{\def\FrameCommand{\hspace{6em}%
%		{\color{myyellow}\vrule width 2pt depth 6pt}\hspace{1em}}%
%	\MakeFramed{\parshape 1 0cm \dimexpr\textwidth-6em\relax\FrameRestore}\vskip2pt%
%}
%{\endMakeFramed}

%\titlecontents{chapter}
%[0em]{\vspace*{2\baselineskip}}
%{\parbox{4.5em}{%
%		\hfill\Huge\sffamily\bfseries\color{myred}\thecontentspage}%
%	\vspace*{-2.3\baselineskip}\leftbar\textsc{\small\chaptername~\thecontentslabel}\\\sffamily}
%{}{\endleftbar}
%\titlecontents{section}
%[8.4em]
%{\sffamily\contentslabel{3em}}{}{}
%{\hspace{0.5em}\nobreak\itshape\color{myred}\contentspage}
%\titlecontents{subsection}
%[8.4em]
%{\sffamily\contentslabel{3em}}{}{}  
%{\hspace{0.5em}\nobreak\itshape\color{myred}\contentspage}



%%%%%%%%%%%%%%%%%%%%%%%%%%%%%%%%%%%%%%%%%%%
% TABLE OF CONTENTS
%%%%%%%%%%%%%%%%%%%%%%%%%%%%%%%%%%%%%%%%%%%

\usepackage{tikz}
\definecolor{doc}{RGB}{0,60,110}
\usepackage{titletoc}
\contentsmargin{0cm}
\titlecontents{chapter}[3.7pc]
{\addvspace{30pt}%
	\begin{tikzpicture}[remember picture, overlay]%
		\draw[fill=doc!60,draw=doc!60] (-7,-.1) rectangle (-0.7,.5);%
		\pgftext[left,x=-3.6cm,y=0.2cm]{\color{white}\Large\sc\bfseries Chapter\ \thecontentslabel};%
	\end{tikzpicture}\color{doc!60}\large\sc\bfseries}%
{}
{}
{\;\titlerule\;\large\sc\bfseries Page \thecontentspage
	\begin{tikzpicture}[remember picture, overlay]
		\draw[fill=doc!60,draw=doc!60] (2pt,0) rectangle (4,0.1pt);
	\end{tikzpicture}}%
\titlecontents{section}[3.7pc]
{\addvspace{2pt}}
{\contentslabel[\thecontentslabel]{2pc}}
{}
{\hfill\small \thecontentspage}
[]
\titlecontents*{subsection}[3.7pc]
{\addvspace{-1pt}\small}
{}
{}
{\ --- \small\thecontentspage}
[ \textbullet\ ][]

\makeatletter
\renewcommand{\tableofcontents}{%
	\chapter*{%
	  \vspace*{-20\p@}%
	  \begin{tikzpicture}[remember picture, overlay]%
		  \pgftext[right,x=15cm,y=0.2cm]{\color{doc!60}\Huge\sc\bfseries \contentsname};%
		  \draw[fill=doc!60,draw=doc!60] (13,-.75) rectangle (20,1);%
		  \clip (13,-.75) rectangle (20,1);
		  \pgftext[right,x=15cm,y=0.2cm]{\color{white}\Huge\sc\bfseries \contentsname};%
	  \end{tikzpicture}}%
	\@starttoc{toc}}
\makeatother

\newcommand{\mytitlea}[4]{
	\begin{tikzpicture}[remember picture,overlay]
		%%%%%%%%%%%%%%%%%%%% Background %%%%%%%%%%%%%%%%%%%%%%%%
		\fill[orange] (current page.south west) rectangle (current page.north east);
		
		
		
		
		%%%%%%%%%%%%%%%%%%%% Background Polygon %%%%%%%%%%%%%%%%%%%%
		
		\foreach \i in {2.5,...,22}
		{
			\node[rounded corners,orange!60,draw,regular polygon,regular polygon sides=6, minimum size=\i cm,ultra thick] at ($(current page.west)+(2.5,-5)$) {} ;
		}
		
		\foreach \i in {0.5,...,22}
		{
			\node[rounded corners,orange!60,draw,regular polygon,regular polygon sides=6, minimum size=\i cm,ultra thick] at ($(current page.north west)+(2.5,0)$) {} ;
		}
		
		\foreach \i in {0.5,...,22}
		{
			\node[rounded corners,orange!90,draw,regular polygon,regular polygon sides=6, minimum size=\i cm,ultra thick] at ($(current page.north east)+(0,-9.5)$) {} ;
		}
		
		
		\foreach \i in {21,...,6}
		{
			\node[orange!85,rounded corners,draw,regular polygon,regular polygon sides=6, minimum size=\i cm,ultra thick] at ($(current page.south east)+(-0.2,-0.45)$) {} ;
		}
		
		
		%%%%%%%%%%%%%%%%%%%% Title of the Report %%%%%%%%%%%%%%%%%%%% 
		\node[left,black,minimum width=0.625*\paperwidth,minimum height=3cm, rounded corners] at ($(current page.north east)+(0,-9.5)$)
		{
			{\fontsize{25}{30} \selectfont \bfseries #1}
		};
		
		%%%%%%%%%%%%%%%%%%%% Subtitle %%%%%%%%%%%%%%%%%%%% 
		\node[left,black,minimum width=0.625*\paperwidth,minimum height=2cm, rounded corners] at ($(current page.north east)+(0,-11)$)
		{
			{\huge \textit{#2}}
		};
		
		%%%%%%%%%%%%%%%%%%%% Author Name %%%%%%%%%%%%%%%%%%%% 
		\node[left,black,minimum width=0.625*\paperwidth,minimum height=2cm, rounded corners] at ($(current page.north east)+(0,-13)$)
		{
			{\Large \textsc{#3}}
		};
		
		%%%%%%%%%%%%%%%%%%%% Year %%%%%%%%%%%%%%%%%%%% 
		\node[rounded corners,fill=orange!70,text =black,regular polygon,regular polygon sides=6, minimum size=2.5 cm,inner sep=0,ultra thick] at ($(current page.west)+(2.5,-5)$) {\LARGE \bfseries #4};
		
	\end{tikzpicture}
}
\newcommand{\mytitleb}[4]{\begin{tikzpicture}[overlay,remember picture]
		
		% Background color
		\fill[
		black!2]
		(current page.south west) rectangle (current page.north east);
		
		% Rectangles
		\shade[
		left color=Dandelion, 
		right color=Dandelion!40,
		transform canvas ={rotate around ={45:($(current page.north west)+(0,-6)$)}}] 
		($(current page.north west)+(0,-6)$) rectangle ++(9,1.5);
		
		\shade[
		left color=lightgray,
		right color=lightgray!50,
		rounded corners=0.75cm,
		transform canvas ={rotate around ={45:($(current page.north west)+(.5,-10)$)}}]
		($(current page.north west)+(0.5,-10)$) rectangle ++(15,1.5);
		
		\shade[
		left color=lightgray,
		rounded corners=0.3cm,
		transform canvas ={rotate around ={45:($(current page.north west)+(.5,-10)$)}}] ($(current page.north west)+(1.5,-9.55)$) rectangle ++(7,.6);
		
		\shade[
		left color=orange!80,
		right color=orange!60,
		rounded corners=0.4cm,
		transform canvas ={rotate around ={45:($(current page.north)+(-1.5,-3)$)}}]
		($(current page.north)+(-1.5,-3)$) rectangle ++(9,0.8);
		
		\shade[
		left color=red!80,
		right color=red!80,
		rounded corners=0.9cm,
		transform canvas ={rotate around ={45:($(current page.north)+(-3,-8)$)}}] ($(current page.north)+(-3,-8)$) rectangle ++(15,1.8);
		
		\shade[
		left color=orange,
		right color=Dandelion,
		rounded corners=0.9cm,
		transform canvas ={rotate around ={45:($(current page.north west)+(4,-15.5)$)}}]
		($(current page.north west)+(4,-15.5)$) rectangle ++(30,1.8);
		
		\shade[
		left color=RoyalBlue,
		right color=Emerald,
		rounded corners=0.75cm,
		transform canvas ={rotate around ={45:($(current page.north west)+(13,-10)$)}}]
		($(current page.north west)+(13,-10)$) rectangle ++(15,1.5);
		
		\shade[
		left color=lightgray,
		rounded corners=0.3cm,
		transform canvas ={rotate around ={45:($(current page.north west)+(18,-8)$)}}]
		($(current page.north west)+(18,-8)$) rectangle ++(15,0.6);
		
		\shade[
		left color=lightgray,
		rounded corners=0.4cm,
		transform canvas ={rotate around ={45:($(current page.north west)+(19,-5.65)$)}}]
		($(current page.north west)+(19,-5.65)$) rectangle ++(15,0.8);
		
		\shade[
		left color=OrangeRed,
		right color=red!80,
		rounded corners=0.6cm,
		transform canvas ={rotate around ={45:($(current page.north west)+(20,-9)$)}}] 
		($(current page.north west)+(20,-9)$) rectangle ++(14,1.2);
		
		% Year
		\draw[ultra thick,gray]
		($(current page.center)+(5,2)$) -- ++(0,-3cm) 
		node[
		midway,
		left=0.25cm,
		text width=5cm,
		align=right,
		black!75
		]
		{
			{\fontsize{25}{30} \selectfont \bf  Lecture\\[10pt] Notes}
		} 
		node[
		midway,
		right=0.25cm,
		text width=6cm,
		align=left,
		orange]
		{
			{\fontsize{72}{86.4} \selectfont #4}
		};
		
		% Title
		\node[align=center] at ($(current page.center)+(0,-5)$) 
		{
			{\fontsize{60}{72} \selectfont {{#1}}} \\[1cm]
			{\fontsize{16}{19.2} \selectfont \textcolor{orange}{ \bf #2}}\\[3pt]
			#3};
\end{tikzpicture}
}
\usepackage{mathcommand}

\tikzcdset{arrow style=tikz,
    squigarrow/.style={
        decoration={
        snake, 
        amplitude=.25mm,
        segment length=2mm
        }, 
        rounded corners=.1pt,
        decorate
        }
    }

\newcommand{\midv}{\,\middle\vert\,}
% define macro
\renewmathcommand{\d}{{\mathrm{d}}}

\begin{document}
\begin{center}
	~\\
	\vspace{6em}
	\textsc{\Huge DIFFERENTIAL MANIFOLD}
	~\\
	\vspace{2.5em}
	{\Large }
	~\\
	\vspace{6em}
	\textsf{Huyi Chen}
	~\\
	\vspace{5in}
	{\large Latest Update: \today}
\end{center}
\newpage
% table of contents
\tableofcontents

% Your document content here
\chapter{Basic Concepts of Manifolds}
\section{Topological Manifold}
\subsection{Definition of Topological Manifold}
\dfn{Topological Manifold}{
    An $n$-dimensional \textbf{topological manifold} is a  second countable Hausdorff topological space that is locally homeomorphic to $n$-dimensional Euclidean space.
}
``Locally homeomorphic to $n$-dimensional Euclidean space" means that for any point $p$ in the topological space $M$, there exists a neighborhood $N(p)$ of $p$ such that $N(p)$ is homeomorphic to the Euclidean space $\mathbb{R}^n$.

\subsection{Coordinate Chart}

\dfn{Coordinate Chart}{
    Let $M$ be topological manifold. If $U\subseteq M$ is an open set containing some point $p\in M$ and
    $$
        \mathbf{x}: U\longrightarrow \mathbf{x} (U)\subseteq \mathbb{R}^n
    $$
    is a homeomorphism, then the pair $(U,\mathbf{x})$ is called a \textbf{coordinate chart} of the topological manifold $M$ at point $p$. 
}
we define projection functions
\[
\begin{aligned}
    p_i:\mathbb{R}^n&\longrightarrow \mathbb{R}\\
    (x_1,x_2,\cdots,x_n)&\longmapsto x_i.
\end{aligned}
\]
\dfn{Coordinate functions}{
    Given a coordinate chart $(U,\mathbf{x})$ of the topological manifold $M$, we can represent $\mathbf{x}$ in its component functions as follows:
    $$
    \mathbf{x}(p)=\left(x^1(p),\cdots,x^n(p)\right),
    $$ 
    where the functions $x^i=p^i\circ \mathbf{x}$ is called the \textbf{$i$-th coordinate function} of the coordinate chart $(U,\mathbf{x})$. And $x^i(p)\in\mathbb{R}$ is called the $i$-th coordinates of point $p$.
}

If $(U,\mathbf{x})$ is a coordinate chart containing point $p$, then $\mathbf{x}^{-1}:\mathbb{R}^n\to M$ is often called the local parameterization of $M$ at point $p$. 

At any point $p$ in the topological manifold $M$, there exists at least one coordinate chart. In fact, let $\eta$ be a homeomorphism from $N(p)$ to $\mathbb{R}^n$. By definition of a neighborhood, there exists an open set $W\subseteq M$ such that $p\in W\subseteq N(p)$. Therefore, by restricting $\eta$ to $W$ we see
$$
\left.\eta\right|_{W}:W\longrightarrow \left.\eta\right|_{W}(W)\subseteq \mathbb{R}^n
$$
is also a homeomorphism. Hence, $(U,\left.\eta\right|_{W})$ is a coordinate chart at point $p$.

\subsection{Atlas}
\dfn{Atlas}{
If $\Sigma=\{(U_\alpha,\mathbf{x}_\alpha)\}_{\alpha\in A}$ is a collection of coordinate charts for the topological manifold $M$ and $\{U_\alpha\}_{\alpha\in
    A}$ is a cover of $M$, then $\Sigma$ is called an \textbf{atlas} of $M$.
}
A topological manifold $M$ can be regarded as a topological space obtained by gluing together many open subsets $U_\alpha(\alpha\in A)$ of Euclidean spaces.

Given an atlas $\Sigma=\{(U_\alpha,\mathbf{x}_\alpha)\}_{\alpha\in
    A}$ of the topological manifold $M$, some coordinate charts in the atlas may have overlapping regions. In other words, some points in $M$ may be covered by multiple coordinate charts. Suppose $(U_\alpha,\mathbf{x}_\alpha)$ and $(U_\beta,\mathbf{x}_\beta)$ have overlapping regions, i.e.,
\[
    \mathbf{x}_\alpha(U_\alpha)^{-1}\cap \mathbf{x}^{-1}_\beta(U_\beta)\ne\varnothing.
\]
Let $V=\mathbf{x}_\alpha(U_\alpha)^{-1}\cap \mathbf{x}^{-1}_\beta(U_\beta)$. Then we can define transition functions
\[
    \mathbf{x}_\beta\circ \mathbf{x}_{\alpha}^{-1}:
    \mathbf{x}_\alpha(V) \longrightarrow
    \mathbf{x}_\beta(V)
\]
and
\[
    \mathbf{x}_{\alpha}\circ
    \mathbf{x}_\beta^{-1}:  \mathbf{x}_\beta(V) \longrightarrow
    \mathbf{x}_\alpha(V).
\]
Since the composition of two homeomorphisms is still a homeomorphism, the above two transition functions are both homeomorphisms. As mentioned earlier, a point $p$ in $M$ may be covered by multiple coordinate charts, meaning that point $p$ may have multiple local coordinates. With the help of transition functions, we can switch between multiple local coordinates of point $p$.

\section{Differentiable Manifold}
\subsection{Motivation: Defining Differentiable Functions}
A topological manifold is a type of topological space that is locally ``very similar" to Euclidean space. Here, ``very similar" refers to a consistent topological structure. Since topological manifolds have a topology that locally behaves like the Euclidean topology, various techniques developed for the Euclidean topology can be applied to topological manifolds. 

As is well known, Euclidean spaces also have many structures more refined than topology, such as the ability to define differentiable functions and compute derivatives. If a topological manifold locally inherits this more refined structure, then, similarly, we can define differentiable functions and compute derivatives on the topological manifold. We will now provide a formal definition of this more refined structure.

Let $M$ be an $n$-dimensional topological manifold and $\Sigma=\{(U_\alpha,\mathbf{x}_\alpha)\}_{\alpha\in
A}$ be an atlas of $M$. Our goal is to define differentiable functions $f:M\to \mathbb{R}^m$ on a topological manifold $M$ in a reasonable way. A natural idea is to locally pull back the domain from a subset of $M$ to $U_\alpha\subset\mathbb{R}^n$ using the coordinate chart $(U_\alpha,\mathbf{x}_\alpha)$. Specifically, we can attempt to define differentiable functions as follows: function $f:M\to \mathbb{R}^m$ is differentiable at point $p$ if and only if for any coordinate chart $(U_\alpha,\mathbf{x}_\alpha)$ covering point $p$, the composite function
\[
    f\circ \mathbf{x}_\alpha^{-1}:\mathbf{x}_\alpha^{-1}\left(U_\alpha\right)\longrightarrow \mathbb{R}^m
\]
is differentiable at point $\mathbf{x}^{-1}_\alpha(p)$.

\subsection{Compatible Atlas}
This definition is theoretically valid but can be cumbersome in practice. To prove the differentiability of $f$ at point $p$, we need to verify differentiablity of $f\circ \mathbf{x}_\alpha^{-1}$ for all coordinate charts $(U_\alpha,\mathbf{x}_\alpha)$ covering point $p$. 

In fact, if we only consider so-called compatible atlases, the situation becomes much simpler. In this case, we only need to verify one arbatrary 
coordinate chart covering point $p$. The following definition makes this idea precise.

\dfn{Compatible Coordinate Charts and Compatible Atlas}{Two coordinate charts $(U_\alpha,\mathbf{x}_\alpha)$, $(U_\beta,\mathbf{x}_\beta)$ are called \textbf{compatible} if
\[
    U_\alpha\cap U_\beta\ne\varnothing\implies\mathbf{x}^{-1}_\beta\circ
\mathbf{x}_{\alpha},\;\mathbf{x}^{-1}_{\alpha}\circ \mathbf{x}_\beta
\text{ are differentiable}.
\]
If any two coordinate charts in an atlas are compatible, the atlas is called a \textbf{compatible atlas}. 
}

Suppose $(U_\alpha,\mathbf{x}_\alpha)$, $(U_\beta,\mathbf{x}_\beta)$ are two coordinate charts covering point $p$. Notice that
$$
\begin{aligned}
f\circ \mathbf{x}_\beta^{-1} &= (f\circ \mathbf{x}^{-1}_\alpha)\circ(\mathbf{x}_\alpha\circ \mathbf{x}^{-1}_{\beta}),\\
f\circ \mathbf{x}_\alpha^{-1} &= (f\circ \mathbf{x}^{-1}_\beta)\circ(\mathbf{x}_\beta\circ \mathbf{x}^{-1}_{\alpha}).
\end{aligned}
$$
We see that $f \circ \mathbf{x}_\beta^{-1}$ is differentiable if and only if $f \circ \mathbf{x}_\alpha^{-1}$ is differentiable. Therefore, if the given atlas is compatible, our definition of differentiable functions can be written as: a function $f: M \to \mathbb{R}^m$ is differentiable at point $p$ if and only if there exists a coordinate chart $(U_\alpha, \mathbf{x}_\alpha)$ covering point $p$ such that the composite function $f\circ \mathbf{x}_\alpha^{-1}$ is differentiable at point $\mathbf{x}^{-1}_\alpha(p)$.

\subsection{Differential Structure and Differentiable Manifold}
Compatible atlases can be maximized.
\dfn{Maximal Atlas of a Compatible Atlas}{
    Let $\Sigma=\{(U_\alpha,\mathbf{x}_\alpha)\}_{\alpha\in A}$ be a compatible atlas of the topological manifold $(M, \tau)$, then
\[
\Sigma_{\max}=\left\{(U_\alpha,\mathbf{x}_\alpha)\mid(U_\alpha,\mathbf{x}_\alpha) \text{ is compatible with every coordinate chart in $\Sigma$}\right\}
\]
is called the \textbf{maximal atlas} of $\Sigma$.
} 
\dfn{Differential Structure and Differentiable Manifold}{
A maximal compatible atlas $\Sigma_{\max}$ on the topological manifold $(M, \tau)$ is called a \textbf{differential structure} of $(M, \tau)$. A topological manifold equipped with a differential structure is called a \textbf{differentiable manifold}, denoted as the triplet $(M, \tau, \Sigma_{\max})$.
}
 Unless otherwise specified, when we talk about coordinate charts of differentiable manifolds, we always assume that the coordinate charts are taken from the differentiable manifold's own atlas.

By replacing ``differentiable" with ``$C^r$-differentiable" or ``smooth", we can similarly define $C^r$-compatible atlases and smooth compatible atlases, $C^r$-differential structures and smooth structures, and $C^r$-differentiable manifolds and smooth manifolds. In the following, we will mainly discuss smooth manifolds, but the related concepts can be extended to differential manifolds and $C^r$-differential manifolds in a similar way.

\ex{Finite-Dimensional Vector Spaces}{
    Let $V$ be a finite-dimensional vector space over $\mathbb{R}$ or $\mathbb{C}$. Any norm on $V$ induces the same topology on $V$. So we can just select any inner product on $V$ and take the induced topology $\tau$ as the topology of $V$.
    
    Let $\mathcal{B}_\alpha=\{v_1,\cdots,v_n\}$ be a basis of $V$. We can define a coordinate chart on $V$ as follows
    \[
    \begin{aligned}
    \mathbf{x_{\alpha}}: V&\longrightarrow \mathbb{R}^n\\
    \mathbf{x_{\alpha}}(x^iv_i)&=(x^1,\cdots,x^n),
    \end{aligned}
    \]
    Given another basis $\mathcal{B}_\beta=\{w_1,\cdots,w_n\}$ of $V$, we can define another coordinate chart on $V$ as follows
    \[
    \begin{aligned}
    \mathbf{x_{\beta}}: V&\longrightarrow \mathbb{R}^n\\
    \mathbf{x_{\beta}}(x^iw_i)&=(x^1,\cdots,x^n),
    \end{aligned}
    \]
    and check that $\mathbf{x}^{-1}_\beta\circ \mathbf{x}_\alpha$ is differentiable. In this way, we can define a compatible atlas on $V$ by taking all possible bases of $V$
    \[
    \Sigma=\{(V,\mathbf{x}_\alpha)\mid \mathcal{B}_\alpha \text{ is a basis of $V$}\}.    
    \]
    Therefore, $(V,\tau,\Sigma_{\max})$ is a differentiable manifold. And the corresponding smooth structure is called the \textbf{standard smooth structure} on $V$.
}

\ex{Spheres as Smooth Manifolds}{
    Let $S^n=\{(x_1,\cdots,x_{n+1})\in\mathbb{R}^{n+1}\mid\|(x_1,\cdots,x_{n+1})\|=1\}$ be the $n$-dimensional sphere. We can define two coordinate charts on $S^n$ as follows:
    \[
    \begin{aligned}
    \mathbf{x}_1: S^n\setminus\{p_N\}&\longrightarrow \mathbb{R}^n\\
    \mathbf{x}_1(x_1,\cdots,x_{n+1})&=\left(\frac{x_1}{1-x_{n+1}},\frac{x_2}{1-x_{n+1}},\cdots,\frac{x_n}{1-x_{n+1}}\right),
    \end{aligned}
    \]
    and
    \[
    \begin{aligned}
    \mathbf{x}_2: S^n\setminus\{p_S\}&\longrightarrow \mathbb{R}^n\\
    \mathbf{x}_2(x_1,\cdots,x_{n+1})&=\left(\frac{x_1}{1+x_{n+1}},\frac{x_2}{1+x_{n+1}},\cdots,\frac{x_n}{1+x_{n+1}}\right),
    \end{aligned}
    \]
    where $p_N=(0,0,\cdots,0,1)$ and $p_S=(0,0,\cdots,0,-1)$. It is easy to verify that $\mathbf{x}_1$ and $\mathbf{x}_2$ are homeomorphisms. Therefore, $(S^n,\tau,\Sigma)$ is a topological manifold, where $\tau$ is the topology induced by the Euclidean topology on $\mathbb{R}^{n+1}$ and $\Sigma=\{(\mathbb{R}^n,\mathbf{x}_1),(\mathbb{R}^n,\mathbf{x}_2)\}_{\max}$ is an atlas of $S^n$. We can verify that $\Sigma$ is a compatible atlas. Therefore, $(S^n,\tau,\Sigma_{\max})$ is a differentiable manifold.
}

\dfn{Open Submanifolds}{
    Let $(M,\tau,\Sigma)$ be a smooth manifold, and $N$ be a open subset of $M$. If we endow $N$ with the subspace topology $\left.\tau\right|_N=\{N\cap U\mid U\in\tau\}$ and an atlas 
    $$
    \Sigma_N=\{(U\cap N,\left.\mathbf{x}\right|_{U\cap N})\mid (U,\mathbf{x})\in\Sigma\}_{\max},
    $$ 
    then $(N,\left.\tau\right|_N,\Sigma_N)$ is a smooth manifold. We call $N$ an \textbf{open submanifold} of $M$.
}


\chapter{Smooth Map}
\section{Definition}
One of the motivations for defining a smooth (differentiable) manifold is to define smooth (differentiable) functions. Let $(M,\tau,\Sigma)$ be a smooth manifold, and a function $f:M\to \mathbb{R}^n$ is smooth at point $p$, if and only if there exists a coordinate chart $(U_\alpha,\mathbf{x}_\alpha)\in\Sigma$ covering point $p$, such that the composite function
$$
f\circ \mathbf{x}^{-1}_\alpha:\mathbf{x}^{-1}_\alpha(U_\alpha)\longrightarrow \mathbb{R}^m
$$
is smooth at point $\mathbf{x}_\alpha(p)$.

Furthermore, we can also define smooth maps between smooth manifolds $(M_1,\tau_1,\Sigma_1), (M_2,\tau_2,\Sigma_2)$. To define the smoothness of mapping $f:M_1\to M_2$ at point $p$, in addition to pulling back the domain from a subset of $M_1$ to $U_\alpha\subset\mathbb{R}^n$, we also need to push forward the codomain from $M_2$ to $V_\beta\subset\mathbb{R}^m$.
\dfn{Smooth Map}{
    The mapping $f:M_1\to M_2$ is a \textbf{smooth map} at point $p$, if there exists a coordinate chart $(U_\alpha,\mathbf{x}_\alpha)\in\Sigma_1$ 
    covering point $p$ and a coordinate chart $(V_\beta,\mathbf{y}_\beta)\in\Sigma_2$ covering point $f(p)$, such that the composite function
    \[
    \mathbf{y}_\beta \circ f\circ \mathbf{x}_\alpha^{-1}:\mathbf{x}_\alpha^{-1}(U_\alpha)\longrightarrow V_\beta
    \]
    is smooth at point $\mathbf{x}_\alpha(p)$.
}
Notice that
\[
    f = \mathbf{y}^{-1}_\beta\circ(\mathbf{y}_\beta \circ f\circ \mathbf{x}^{\hspace{2pt}-1}_\alpha)\circ\mathbf{x}_\alpha
\]
we see that $f$ being smooth at point $p$ implies that $f$ is continuous at point $p$.
Moreover, note that 
\[
    \tilde{\mathbf{y}}_\beta \circ f\circ \tilde{\mathbf{x}}_\alpha^{-1}=(\tilde{\mathbf{y}}_\beta\circ\mathbf{y}_\beta^{-1}) \circ(\mathbf{y}_\beta \circ f\circ \mathbf{x}_\alpha^{-1})\circ(\mathbf{x}_\alpha\circ \tilde{\mathbf{x}}_\alpha^{-1}),
\]
where $\tilde{\mathbf{x}}_\alpha$ and $\tilde{\mathbf{y}}_\beta$ are two other coordinate charts covering point $p$ and $f(p)$ respectively. It is clear that the smoothness of $f:M_1\to M_2$ at point $p$ does not depend on the choice of coordinate charts.

If the mapping $f:M_1\to M_2$ is smooth in an open set $U\subset M_1$, then we say that $f$ is smooth on $U$. If $f$ is smooth at every point in $M_1$, then we say that $f$ is a smooth map.

\dfn{Diffeomorphism}{
    Given two smooth (differentiable) manifolds $(M_1,\tau_1,\Sigma_1), (M_2,\tau_2,\Sigma_2)$, a smooth (differentiable) mapping $f:M_1\to M_2$ is called a \textbf{diffeomorphism} if $f$ is bijective and $f^{-1}$ is also smooth (differentiable).
}

It is a worthwhile problem to classify the smooth structures that may exist on a topological manifold through diffeomorphisms. Some literature, when defining smooth structures, directly mods out the diffeomorphism equivalence relations on top of our definition of smooth structures. To avoid confusion, we refer to the equivalence classes of smooth structures modding out diffeomorphisms as smooth types, which is defined as follows.

\dfn{Smooth Type}{
    Given a topological manifold $(M,\tau)$, the equivalence class of smooth structures modding out diffeomorphism equivalence relations is called the \textbf{smooth type} of $M$. That is, if $(M_1,\tau_1,\Sigma_1)$ is diffeomorphic to $(M_2,\tau_2,\Sigma_2)$, then $(M_1,\tau_1,\Sigma_1)$ and $(M_2,\tau_2,\Sigma_2)$ have the same smooth type.
}

Hassler Whitney's results show that for any $r\ge 1$, every $C^r$ differential structure contains some smooth structures, and all of these smooth structures are diffeomorphic, i.e. corresponding to one smooth type. Therefore, we can also classify $C^r$ differentiable structures by smooth types. This is why we will only focus on smooth manifolds and smooth functions later on. 

For topological manifolds of dimension $\le 3$, there is only one smooth type. For compact topological manifolds with dimension $\ge 3$, the number of smooth types is finite. When $n\ne 4$, $\mathbb{R}^n$ only has one smooth type. However, $\mathbb{R}^4$ has uncountably many smooth types.

The category of smooth manifolds is denoted as $\mathsf{Diff}^\infty$, whose objects are smooth manifolds and morphisms are smooth maps.

\section{Tangent Vector and Tangent Space}

Now we have defined smooth maps between smooth manifolds, we have yet to explain how to differentiate such smooth maps - which is one of our motivations for defining smooth (differentiable) manifolds. If we follow the idea of defining smooth maps, using coordinate charts to convert smooth maps into smooth functions from $U_\alpha\in\mathbb{R}^n$ to $\mathbb{R}^m$, we will find that different coordinate charts yield different smooth functions and, naturally, different differentials. A simplest example is a smooth real-valued function $f:\mathbb{R}\to\mathbb{R}$. When selecting different coordinate charts $(U_\alpha,\mathbf{x}_\alpha),(V_\beta,\mathbf{y}_\beta)$ that contain a point $p$, the differentials as linear transformations on $\mathbb{R}$
$$
\d(\mathbf{y}_\beta \circ f\circ \mathbf{x}_\alpha^{-1})|_p:\mathbb{R}\longrightarrow\mathbb{R}
$$
may differ in general. However, we know these differentials cannot be entirely unrelated, as any differences between them should only arise from the differential of the transition function. A similar situation also occurs in linear algebra. With different bases chosen, a linear transformation can induce different transformations from the coordinate space $\mathbb{R}^n$ to the coordinate space $\mathbb{R}^m$. The only discrepancy between these transformations arises from the change of basis. Therefore, we can imagine that the true ``differential" of a smooth map might also be a linear mapping between general linear spaces. Only by choosing specific coordinate charts do we obtain the linear mapping between Euclidean spaces $\mathbb{R}^n$ and $\mathbb{R}^m$. In fact, the ``general linear spaces" here are the tangent spaces we are about to define.

\subsection{First Definition: Equivalence Class of Curves}
Consider an intuitive example, a smooth surface in $\mathbb{R}^3$ of dimension $2$. At any point $p$ on the surface, we can construct a tangent plane, and the vector with its starting point at $p$ and its endpoint in the tangent plane is called a tangent vector. Therefore, the tangent plane of the surface at point $p$ can be regarded as a linear space $\mathbb{R}^2$ composed of tangent vectors.

For any tangent vector, it can be viewed as the tangent vector of a smooth curve $\gamma$ passing through point $p$ on the surface at $p$. If the parameter equation of curve $\gamma$ is $\gamma(t)=(x(t),y(t),z(t))$ and satisfies $\gamma(t_0)=p$, then its derivative at point $p$
\[
    \gamma'(t_0)=\left(\left.\frac{\d x}{\d t}\right|_{t_0},\left.\frac{\d y}{\d t}\right|_{t_0}, \left.\frac{\d z}{\d t}\right|_{t_0}\right)
\]
is a tangent vector at $p$. Suppose there is another curve $\theta$ passing through point $p$ on the surface. If $\theta$ and $\gamma$ have the same derivative at point $p$, i.e., $\theta'(t_0)=\gamma'(t_0)$, then these two curves actually correspond to the same tangent vector. Therefore, a tangent vector can also be viewed as an equivalence class of curves, and all curves in the class have the same derivative at point $p$. If we can extend the concepts of curves and curve derivatives to general smooth manifolds, we can define tangent vectors and tangent spaces for smooth manifolds in the same way.

Given an $n$-dimensional smooth manifold $(M,\tau,\Sigma)$ and a point $p$ on it, if a smooth map $\gamma:(-\epsilon,\epsilon)\to M$ satisfies $\gamma(0)=p$, then $\gamma$ is called a smooth curve on $M$ passing through point $p$. If we want to ``differentiate" the curve, we must first parameterize it, i.e., select a coordinate chart $(U,\mathbf{x})\in\Sigma$ at point $p$ and differentiate the composite function
\[
    \mathbf{x} \circ \gamma:(-\epsilon,\epsilon)\longrightarrow\mathbb{R}^n.
\]
The result of differentiation is $\left.\frac{\d(\mathbf{x} \circ \gamma)}{\d t}\right|_0$. 
\prop{}{
    If curves $\theta$ and $\gamma$ have the same derivative at point $p$ in a certain coordinate chart $(U_1,\mathbf{x}_1)$, namely
    \[
        \left.\frac{\d (\mathbf{x}_1 \circ \theta)}{\d t}\right|_0=\left.\frac{\d(\mathbf{x}_1 \circ \gamma)}{\d t}\right|_0,
    \]
    then in any coordinate chart $(U_2,\mathbf{x}_2)$ at point $p$, the curves $\theta$ and $\gamma$ also have the same derivative. 
}
\pf{
    According to the chain rule, we have
    \[
        \left.\frac{\d(\mathbf{x}_2 \circ \theta)}{\d t}\right|_0=\left.\frac{\d((\mathbf{x}_2 \circ \mathbf{x}_1^{-1})\circ(\mathbf{x}_1 \circ \theta))}{\d t}\right|_0=\left.\frac{\d(\mathbf{x}_2 \circ \mathbf{x}_1^{-1})}{dx}\right|_p\left.\frac{\d(\mathbf{x}_1 \circ \theta)}{\d t}\right|_0.
    \]
    Similarly, we have
    \[
        \left.\frac{\d(\mathbf{x}_2 \circ \gamma)}{\d t}\right|_0=\left.\frac{\d(\mathbf{x}_2 \circ \mathbf{x}^{-1}_1)}{dx}\right|_p\left.\frac{\d(\mathbf{x}_1 \circ \gamma)}{\d t}\right|_0.
    \]
    Thus we show that
    \[
        \left.\frac{\d(\mathbf{x}_1 \circ \theta)}{\d t}\right|_0=\left.\frac{\d(\mathbf{x}_1 \circ \gamma)}{\d t}\right|_0\implies \left.\frac{\d(\mathbf{x}_2 \circ \theta)}{\d t}\right|_0=\left.\frac{\d(\mathbf{x}_2 \circ \gamma)}{\d t}\right|_0.
    \]
}
We can define that curves $\theta$ and $\gamma$ are equivalent at point $p$ if and only if there exists a coordinate chart $(U,\mathbf{x})$ at point $p$ such that
\[
    \left.\frac{\d(\mathbf{x} \circ \theta)}{\d t}\right|_0=\left.\frac{\d(\mathbf{x} \circ \gamma)}{\d t}\right|_0.
\]
We denote the equivalence class of curve $\gamma$ as $[\gamma]_*$. The set of all smooth curves passing through point $p$ on $M$ under this equivalence relation is the quotient set of the tangent space of $M$ at point $p$, denoted as

\[
    T_p^{1\mathrm{st}}M = \{[\gamma]_*: \gamma \text{ is a smooth curve on } M \text{ passing through } p\}.
\]

\subsection{Second Definition: Functional induces by Curves}
\subsubsection{Motivation}
By regarding tangent vectors as equivalence classes of curves, we have successfully extended the notion of tangent vectors of smooth surfaces in $\mathbb{R}^n$ to general smooth manifolds. Although the definition of tangent vectors via curves is intuitive, demonstrating that the set of all tangent vectors forms a linear space requires defining appropriate addition and scalar multiplication operations on the equivalence classes of curves. 

We may be tempted to define the sum of equivalence classes of curves as the equivalence class of the sum of curves. However, a simple example illustrates that the set of all smooth curves passing through a point $p$ does not form a linear space under such an addition operation. For instance, consider the curves $\gamma$ and $\theta$ such that $\gamma(0)=\theta(0)=p$. It follows that $(\gamma+\theta)(0)=2p\neq p$, indicating that the sum of two curves passing through $p$ may not pass through $p$.

While it is possible to define a novel addition operation for equivalence classes of curves to make them form a linear space, such an operation may prove to be unnatural and inelegant. This observation also suggests that the perspective of equivalence classes of curves may not be the best way to approach the notion of tangent spaces. Instead, a more promising solution may involve re-examining our definition of tangent vectors.

From the definition of tangent vectors, we see there is a mapping from $T_p^{1\mathrm{st}}M$ to $\mathbb{R}^n$
\[
    \mathrm{Drv}_{\mathbf{x}}:[\gamma]_*\longmapsto\left.\frac{\d(\mathbf{x} \circ \gamma)}{\d t}\right|_0.
\]
You can further prove that $\mathrm{Drv}_{\mathbf{x}}$ is actually a bijection. This may suggest defining tangent space as
\[
    \mathrm{Drv}_{\mathbf{x}}\left(T_p^{1\mathrm{st}}M\right)=\left\{\left.\frac{\d(\mathbf{x} \circ \gamma)}{\d t}\right|_0: [\gamma]_*\in T_p^{1\mathrm{st}}M \right\}
\]
which is naturally equipped with a linear space structure. However, this definition is not satisfactory because it is not coordinate-independent. We have already seen earlier that the value of $\left.\frac{\d(\mathbf{x} \circ \gamma)}{\d t}\right|_0$ depends on the choice of coordinate chart $(U,\mathbf{x})$. As a consequence,  there is no canonical way to assign an $n$-dimensional vector to each equivalence class of curves. 

In order to give a coordinate-independent definition of tangent vectors, we consider a mapping like
\[
    \mathrm{Drv}:[\gamma]_*\longmapsto\left.\frac{\d(\cdot \circ \gamma)}{\d t}\right|_0.
\]
Since $\mathbf{x}$ is a smooth map from $M$ to $\mathbb{R}^n$, one might consider letting $\left. \frac{\d(\cdot \circ \gamma)}{\d t} \right|_0$ act on any smooth map $\mathbf{f}:M\to\mathbb{R}^n$. However, it is more economical to let it act on smooth functions $\mathbf{f}:M\to\mathbb{R}$, as the action on a vector-valued function $\mathbf{f}:M\to\mathbb{R}^n$ can be decomposed into actions on its component functions $f_i:M\to\mathbb{R}$.
For a vector-valued function $\mathbf{f} : M \to \mathbb{R}^n$, the $i$-th component function of $\mathbf{f}$ is given by $f_i = p_i \circ \mathbf{f}$. Since
\[
    \left(p_i\circ\left.\frac{\d(\mathbf{f} \circ \gamma)}{\d t}\right)\right|_0=\left.\frac{\d(p_i\circ\mathbf{f} \circ \gamma)}{\d t}\right|_0=\left.\frac{\d(f_i \circ \gamma)}{\d t}\right|_0,
\]
we have
\[
    \left.\frac{\d(\mathbf{f} \circ \gamma)}{\d t}\right|_0=\left(\left.\frac{\d(f_1 \circ \gamma)}{\d t}\right|_0,\left.\frac{\d(f_2 \circ \gamma)}{\d t}\right|_0,\cdots,\left.\frac{\d(f_n \circ \gamma)}{\d t}\right|_0\right).
\]
\subsubsection{Curves induce functionals}
Let $\mathcal{C}^{\infty}_p(M)$ denote the linear space of functions on $M$ that are smooth at the point $p$. Now we can define the functional induced by $\gamma$ as 
\[
\begin{aligned}
    v_\gamma:=\left.\frac{\d(\cdot \circ
\gamma)}{\d t}\right|_0:\mathcal{C}^{\infty}_p(M)&\longrightarrow \mathbb{R}\\
    f&\longmapsto \left.\frac{\d(f \circ \gamma)}{\d t}\right|_0.
\end{aligned}
\]
We can check that $v_\gamma$ is a linear functional. For any $f_1,f_2\in\mathcal{C}^{\infty}_p(M)$ and $k\in\mathbb{R}$, we have
$$
\left.\frac{\d((f_1+f_2) \circ \gamma)}{\d t}\right|_0=\left.\frac{\d(f_1\circ \gamma+f_2\circ \gamma )}{\d t}\right|_0=\left.\frac{\d(f_1\circ \gamma)}{\d t}\right|_0+\left.\frac{\d(f_2\circ \gamma)}{\d t}\right|_0,
$$

$$
\left.\frac{\d((kf) \circ \gamma)}{\d t}\right|_0=\left.\frac{\d(k(f \circ \gamma))}{\d t}\right|_0=k\left.\frac{\d(f \circ \gamma)}{\d t}\right|_0.
$$
Let $\mathcal{C}^{\infty}_p(M)^*$ denote the dual space of $\mathcal{C}^{\infty}_p(M)$, which is the linear space of all linear functionals on $\mathcal{C}^{\infty}_p(M)$. We can now provide a second coordinate-independent definition of tangent space as follows:
$$
T^{2\mathrm{nd}}_pM=\left\{\left.\frac{\d(\cdot \circ \gamma)}{\d t}\right|_0\in\mathcal{C}^{\infty}_p(M)^*:\gamma\text{ is a smooth curve on $M$ passing through $p$ }\right\}.
$$
\subsubsection{Natural basis}
Next we will introduce natural basis and use it to show that $T^{2\mathrm{nd}}_pM$ is indeed a linear subspace of $\mathcal{C}^{\infty}_p(M)^*$ of dimension $n$.



\lemm{}{
    If we choose a coordinate chart $(U,\mathbf{x})$ of $M$ at $p$, then for any $v_\gamma\in T^{2\mathrm{nd}}_pM$, $f\in\mathcal{C}^{\infty}_p(M)$,
    \[
        v_\gamma(f)=\sum_{i=1}^n v_\gamma(x^i) \left.\frac{\partial\left(f\circ \mathbf{x}^{-1}\right)}{\partial x^i}\right|_{p}.
    \]
}
\pf{
We only need to use the pullback of $f$ and pushforward of $\gamma$ with respect to the coordinate chart $(U,\mathbf{x})$, and then apply the chain rule:
\[
    v_\gamma(f)=\left.\frac{\d(f \circ\gamma)}{\d t}\right|_0=\left.\frac{\d((f \circ \mathbf{x}^{-1})\circ (\mathbf{x}\circ \gamma))}{\d t}\right|_0=\sum_{i=1}^n\left.\frac{\partial (f \circ \mathbf{x}^{-1})}{\partial x^i}\right|_{p}\left.\frac{\d({x}^i \circ \gamma)}{\d t}\right|_0.
\]
}
\noindent Define the \textbf{$x_i$-curve} for the coordinate chart $(U,\mathbf{x})$ as
\[
    \begin{aligned}
        \varepsilon_i:(-\epsilon,\epsilon)&\longrightarrow M\\
        t&\longmapsto\mathbf{x}\left(x^1(p),\cdots,{x}^i(p)+t,\cdots,x^n(p)\right).
    \end{aligned}
\]
$v_{\varepsilon_i}$ is called the \textbf{$i$-th coordinate tangent vector} of the coordinate chart $(U,\mathbf{x})$.
We can check that
\[
    v_{\varepsilon_i}\left( x^j\right)=\left.\frac{\d( x^j\circ\varepsilon_i)}{\d t}\right|_0=\delta_{ij}=\begin{cases}
        0,& i\ne j,\\
        1,& i=j.
        \end{cases}
\]
By lemma 2.2.1, we have
\[
    v_{\varepsilon_i}(f)=\sum_{j=1}^n v_{\varepsilon_i}( x^j) \left.\frac{\partial (f\circ \mathbf{x}^{-1})}{\partial x^j}\right|_{p}=\left.\frac{\partial (f\circ \mathbf{x}^{-1})}{\partial x^i}\right|_{p},
\]
which implies 
\[
    v_{\varepsilon_i}=\left.\frac{\d(\cdot \circ\varepsilon_i)}{\d t}\right|_0=\left.\frac{\partial (\cdot \circ \mathbf{x})}{\partial x^i}\right|_{p}.
\]
\dfn{Natural Basis}{
Given a coordinate chart $(U,\mathbf{x})$ of smooth manifold $M$ at point $p$, the coordinate tangent vectors
\[
    (v_{\varepsilon_1},v_{\varepsilon_2},\cdots,v_{\varepsilon_n})=\left(\left.\frac{\partial (\cdot \circ \mathbf{x}^{-1})}{\partial x^1}\right|_{p},\left.\frac{\partial (\cdot \circ \mathbf{x}^{-1})}{\partial x^2}\right|_{p},\cdots,\left.\frac{\partial (\cdot \circ \mathbf{x}^{-1})}{\partial x^n}\right|_{p}\right)
\]
form a basis for the linear space $T_p^{\mathrm{2nd}}M$, called the \textbf{natural basis} of the tangent space $T_p^{\mathrm{2nd}}M$ under the coordinate chart $(U,\mathbf{x})$.

By abuse of notation, we also denote the natural basis by 
\[
    (v_{\varepsilon_1},v_{\varepsilon_2},\cdots,v_{\varepsilon_n})=\left(\left.\frac{\partial }{\partial x^1}\right|_{p},\left.\frac{\partial}{\partial x^2}\right|_{p},\cdots,\left.\frac{\partial }{\partial x^n}\right|_{p}\right)  
\]
}
\pf{
    The linear space spanned by the coordinate tangent vectors is denoted by $\mathop{\mathrm{span}}\limits_{1\le i \le n}(v_{\delta_i})$. Lemma 2.1.1 states that 
    \[
        v_\gamma(f)=\sum_{i=1}^n v_\gamma({x}^i) \left.\frac{\partial (f\circ \mathbf{x})}{\partial x_i}\right|_{p}=\sum_{i=1}^n v_\gamma({x}^i) v_{\varepsilon_i}(f),
    \]
    or equivalently,
    \[
        v_\gamma = \sum_{i=1}^n v_\gamma({x}^i) v_{\varepsilon_i}.
    \]
    That means any vector in $T^{2\mathrm{nd}}_pM$ can be written as a linear combination of coordinate tangent vectors, namely 
    \[
        T^{2\mathrm{nd}}_pM\subseteq
        \mathop{\mathrm{span}}\limits_{1\le i \le
        n}(v_{\varepsilon_i}).
    \]
    Conversely, for any linear combination of coordinate tangent vectors $\sum\limits_{i=1}^n\lambda_iv_{\varepsilon_i}$, we can define a curve
    \[
        \begin{aligned}
            \varepsilon:(-\epsilon,\epsilon)&\longrightarrow M\\
            t&\longmapsto\mathbf{x}(\tilde{x}_1(p)+\lambda_1t,\tilde{x}_2(p)+\lambda_2t,\cdots,\tilde{x}_n(p)+\lambda_nt),
        \end{aligned}
    \]
    such that
    \[
        v_\varepsilon = \sum_{i=1}^n v_\varepsilon({x}^i) v_{\varepsilon_i}=\sum_{i=1}^n\left.\frac{\d({x}^i \circ \varepsilon)}{\d t}\right|_0 v_{\varepsilon_i}=\sum_{i=1}^n\left.\frac{\d(p_i\circ\mathbf{x}^{-1} \circ \varepsilon)}{\d t}\right|_0 v_{\varepsilon_i}=\sum_{i=1}^n\left.\frac{\d({x}^i(p)+\lambda_it)}{\d t}\right|_0 v_{\varepsilon_i}=\sum_{i=1}^n\lambda_iv_{\varepsilon_i}.
    \]
    Thus we have $\mathop{\mathrm{span}}\limits_{1\le i \le n}(v_{\varepsilon_i})\subseteq T^{2\mathrm{nd}}_pM$. Therefore, we show that
    \[
        T^{2\mathrm{nd}}_pM=\mathop{\mathrm{span}}\limits_{1\le i \le n}(v_{\varepsilon_i}),
    \]
    which justifies $T^{2\mathrm{nd}}_pM$ is a linear space.

    Furthermore, we can prove that $v_{\varepsilon_1},v_{\varepsilon_2},\cdots,v_{\varepsilon_n}$ are linearly independent. If
    \[
        \sum_{i=1}^n\lambda_iv_{\varepsilon_i}=0,
    \]
    let the two sides of the above equation act on $ x^j$ for $1\le j\le n$. Then we have
    \[
        \sum_{i=1}^n\lambda_iv_{\varepsilon_i}( x^j)=\sum_{i=1}^n\lambda_i\delta_{ij}=\lambda_j=0,
    \]
    for $1\le j\le n$, which proves that they are linearly independent. Therefore, $v_{\varepsilon_1},v_{\varepsilon_2},\cdots,v_{\varepsilon_n}$ form a basis for $T^{2\mathrm{nd}}_pM$. 
}
\cor{}{
    The dimension of the tangent space $T^{2\mathrm{nd}}_pM$ is $n$, which is the same as the dimension of $M$.
}
\ex{Natural Basis of $\mathbb{R}^n$}{
    Suppose $\mathbb{R}^n$ is a smooth manifold with the standard coordinate chart $(\mathbb{R}^n,\mathrm{id})$. Then the natural basis of the tangent space $T^{2\mathrm{nd}}_p\mathbb{R}^n$ under the coordinate chart $(\mathbb{R}^n,\mathrm{id})$ is
    \[
        (v_{\varepsilon_1},v_{\varepsilon_2},\cdots,v_{\varepsilon_n})=\left(\left.\frac{\partial}{\partial x^1}\right|_{p},\left.\frac{\partial}{\partial x^2}\right|_{p},\cdots,\left.\frac{\partial }{\partial x^n}\right|_{p}\right),
    \]
    where 
    \[
        \begin{aligned}
            \left.\frac{\partial}{\partial x^i}\right|_{p}:C^\infty(\mathbb{R}^n)&\longrightarrow\mathbb{R}\\
            f&\longmapsto\left.\frac{\partial f}{\partial x^i}\right|_{p}
        \end{aligned}
    \]
    is the $i$-th partial derivative of $f$ at $p$.
}

\subsection{Third Definition: Language of Sheaf Theory}
We can also use the language of sheaf theory to define tangent vectors and tangent spaces. Sheaf theory is a more abstract theory that is not only applicable to differential geometry, but also to more general geometry.

A \textbf{presheaf} is essentially a contravariant functor. A set presheaf is a contravariant functor $\mathcal{F}:\mathsf{C}^{\mathrm{op}}\to\mathsf{Set}$ from a category $\mathsf{C}$ to the category of sets $\mathsf{Set}$. A group presheaf is a contravariant functor $\mathcal{F}:\mathsf{C}^{\mathrm{op}}\to\mathsf{Ab}$ from a category $\mathsf{C}$ to the category of abelian groups $\mathsf{Ab}$. Given an object $X$ in $\mathsf{C}$, an element of $\mathcal{F}(X)$ is called a \textbf{section} of $\mathcal{F}$ over $X$.

When studying a smooth manifold $M$, we choose $\mathsf{C}$ to be the category whose objects are open sets in $M$, and whose morphisms are inclusions of open sets. We consider a presheaf on an open set $X$ to be the collection of all smooth real-valued functions defined on $X$, denoted $C^\infty(X;M\to\mathbb{R})$, which we can also write as $C^\infty(X)$ for short.

The functor $C^\infty(\cdot)$ can be illustrated as follows:
\begin{equation*}
    \begin{tikzcd}
        \mathsf{Open}_M&[-25pt] & [+10pt] & [-25pt]\mathbb{R}\text{-}\mathsf{CAlg}\\[-14pt] 
        X  \arrow[dd, "\iota"{name=L, left}] & & &  C^\infty(X) \\ [-10pt] 
                                        &  \phantom{.}\arrow[r, "C^\infty", squigarrow]&\phantom{.}  &   \\[-10pt] 
        Y & & & C^\infty(Y)\arrow[uu, "\iota^*"']
    \end{tikzcd}
\end{equation*}  
Depending on our needs, $C^\infty(\cdot)$ can be a presheaf of real vector spaces, a presheaf of rings, or a presheaf of commutative real algebras.

Given a point $p$ in $M$, we denote by
\[
    \widetilde{\mathcal{C}}^\infty_p=\{(\varphi,X):p\in X,\varphi\in C^\infty(X)\}
\]
the set of smooth real-valued functions defined on an open set containing $p$. We define $C^\infty_p=\widetilde{\mathcal{C}}_p/\overset{p}\sim$ where $\overset{p}\sim$ is the equivalence relation defined as follows:
\[
    (\varphi_1,X_1)\overset{p}\sim (\varphi_2,X_2)\iff\text{there exists an open set }X_3\subset X_1\cap X_2,\text{such that } \varphi_1|_{X_3}=\varphi_2|_{X_3}.
\]
Or in a language of filtered limit,
\[
    C^\infty_p=\varinjlim_{p\in X}C^\infty(X).
\]
We call $C^\infty_p$ the stalk of smooth functions at the point $p$. It can be easily verified that the smooth function stalk is a commutative real algebra. The equivalence class of a smooth function $\varphi$ is denoted as $[\varphi]_p$ and is called the germ of $\varphi$ at point $p$. Since all smooth functions within a germ have identical local properties at point $p$, it is natural to identify them and not to consider their differences. 

Using the language of function germs, we can rewrite the second definition. The functional $\left.\frac{d}{\d t}(\cdot \circ \gamma)\right|_0$ was originally applied to a smooth function $f$ defined on the entire manifold and smooth at point $p$, but the properties of $f$ outside some neighborhood of point $p$ are actually irrelevant, and even whether it is defined or not is not important. Therefore, we can assume that $f$ is defined in some open neighborhood of point $p$, and then let $\left.\frac{d}{\d t}(\cdot \circ \gamma)\right|_0$ act directly on the germ of the smooth function $f$ at point $p$, denoted as $[f]_p$, which is defined as
\[
    \begin{aligned}
        \left.\frac{\d(\cdot \circ \gamma)}{\d t}\right|_0:C^\infty_p&\longrightarrow \mathbb{R}\\
        [f]_p&\longmapsto \left.\frac{\d(f \circ \gamma)}{\d t}\right|_0
    \end{aligned}
\]
And correspondently, the tangent space $T^{2\mathrm{nd}}_pM$ can be defined as
\[
    T^{2\mathrm{nd}}_pM=\left\{\left.\frac{\d(\cdot \circ \gamma)}{\d t}\right|_0\in \left(C^{\infty}_p\right)^*:\gamma\text{ is a smooth curve on $M$ passing through $p$ }\right\}.
\]
Furthermore, we can provide a concise definition of a tangent vector without the use of curves, through an axiomatic approach. A tangent vector at a point $p$ of a manifold $M$ is a linear functional defined on $C^\infty_p$ that satisfies the Leibniz rule. In a formal manner, this can be expressed as follows:
\dfn{Tangent Space}{
    Given a smooth manifold $M$ and a point $p$ on it. A function $v: C^\infty_p \to \mathbb{R}$ is said to be a \textbf{tangent vector} of manifold $M$ at point $p$ if it satisfies the following conditions for any $\lambda_1, \lambda_2 \in \mathbb{R}$ and any $[f]_p, [g]_p \in C^\infty_p$:
    \begin{enumerate}
        \item Linearity: $v(\lambda_1[f]_p + \lambda_2[g]_p) = \lambda_1v([f]_p) + \lambda_2v([g]_p)$,
        \item Leibniz rule: $v([f]_p[g]_p) = v([f]_p)[g]_p+[f]_pv([g]_p)$.
    \end{enumerate}
    The set of tangent vectors of a manifold $M$ at point $p$ forms a linear space, denoted by $T_p^{\mathrm{3th}}M$, and is called the \textbf{tangent space} of $M$ at point $p$.
}
Due to the equivalence of the three distinct definitions of tangent space, we will adopt the notation $T_pM$ to refer to the tangent space of a manifold $M$ at a given point $p$, without explicitly specifying the definition being used.



\subsection{Tagent Bundle}

\dfn{Tangent Bundle}{
    Let $(M, \tau, \Sigma)$ be a smooth manifold with atlas $\Sigma=\{(U_\alpha,\mathbf{x}_\alpha)\}_{\alpha\in A}$. The \textbf{tangent bundle} of $M$ is the disjoint union of the tangent spaces of $M$ at all points, denoted by $TM$:
    \[
        TM=\bigsqcup_{p\in M}T_pM=\{(p,v):p\in M,v\in T_pM\}.
    \]
    The projection map $\pi:TM\to M$ is defined as $\pi(p,v)=p$. For any $(U_\alpha,\mathbf{x}_\alpha)\in \Sigma$, we can define a mapping
    \begin{align*}
        \varphi_\alpha:\pi^{-1}\left(U_\alpha\right)&\longrightarrow \mathbb{R}^{2n}\\
        (p,v)&\longmapsto \left(x^1(p),\cdots,x^n(p),v^1,\cdots,v^n\right)
    \end{align*}
    where $v=v^i\left.\frac{\partial}{\partial x^i}\right|_p$. In fact, $\varphi_\alpha$ is a bijection and its inverse can be explicitly written as
    \[
        \varphi_\alpha^{-1}(x^1,\cdots,x^n,v^1,\cdots,v^n)=\left(\mathbf{x}^{-1}_\alpha(x^1,\cdots,x^n),v^i\left.\frac{\partial}{\partial x^i}\right|_{(x^1,\cdots,x^n)}\right).
    \]


     The topology of $TM$ is defined as 
    \[
        \tau_{TM}=\left\{S\in 2^{TM}\midv\forall \alpha, \,\varphi_\alpha(S\cap \pi^{-1}(U_\alpha))\in\tau\right\}.
    \]
    We can check that $\varphi_\alpha$ is a homeomorphism between $\pi^{-1}\left(U_\alpha\right)$ and $\mathbb{R}^{2n}$ and therefore serve as charts for $TM$. Furthermore, $TM$ is a smooth manifold with the following atlas
    \[
        \Sigma_{TM}=\{(\pi^{-1}\left(U_\alpha\right),\varphi_\alpha)\}_{\alpha\in A}.
    \]
}

\dfn{Smooth Vector Bundle}{
    Given a vector bundle $(E,\pi,M)$, where $E$ is total space, $M$ is base space, and $\pi:E\to M$ is the bundle projection, we say that $(E,\pi,M)$ is a \textbf{smooth vector bundle} if $E$ and $M$ are smooth manifolds, $\pi$ is a smooth map, and the local trivializations are can be chosen to be diffeomorphisms.
}

\prop{}{
    The tangent bundle $(TM,\pi,M)$ is a smooth vector bundle.
}

\dfn{Vector Field}{
    A \textbf{vector field} on a smooth manifold $M$ is a continuous section of the tangent bundle projection $\pi:TM\to M$. In other words, a vector field is a continuous mapping 
    \[
        \begin{aligned}
            X:M&\longrightarrow TM\\
            p&\longmapsto X_p
        \end{aligned}
    \]
    such that $\pi\circ X=\mathrm{id}_M$.
}

\dfn{Smooth Vector Field}{
    Let $M$ be a smooth manifold. A vector field $X:M\to TM$ is said to be \textbf{smooth} if it is a smooth map. The set of all smooth vector fields on $M$ is denoted by $\mathfrak{X}(M)$.
}

\ex{Coordinate Vector Fields}{
    Let $(U,\mathbf{x})$ be a coordinate chart of $M$. Then the \textbf{$i$-th coordinate vector field} defined as follows
    \[
        \begin{aligned}
            \frac{\partial}{\partial x^i}:U&\longrightarrow TU\\
            p&\longmapsto \frac{\partial}{\partial x^i}\bigg|_p
        \end{aligned}
    \]
    is a smooth vector field on $U$.
}

\prop{$\mathfrak{X}(M)$ as a $C^\infty(M)$-Module}{
    Let $M$ be a smooth manifold. Then $\mathfrak{X}(M)$ is a $C^\infty(M)$-module with
    \begin{itemize}
        \item addition: $(X+Y)_p=X_p+Y_p$ for any $p\in M$ and $X,Y\in\mathfrak{X}(M)$,
        \item scalar multiplication: 
    \[
        \begin{aligned}
            C^\infty(M)\times \mathfrak{X}(M)&\longrightarrow \mathfrak{X}(M)\\
            (f,X)&\longmapsto fX
        \end{aligned}
    \]
    where $fX:M\to TM$ is defined as $\left(fX\right)_p=f(p)X_p$ for any $p\in M$.\end{itemize}
}

\dfn{Derivation on an $R$-algebra}{
    Let $R$ be a commutative ring. A \textbf{derivation} on an $R$-algebra $A$ is a map $D:A\to A$ such that
    \begin{enumerate}[label=(\roman*)]
        \item $R$-linear: $D(\lambda a+\mu b)=\lambda D(a)+\mu D(b)$ for any $\lambda,\mu\in R$ and $a,b\in A$,
        \item Leibniz rule: $D(ab)=D(a)b+aD(b)$ for any $a,b\in A$.
    \end{enumerate}
}

\dfn{Derivation on $C^\infty(M)$}{
    Let $M$ be a smooth manifold. A \textbf{derivation} on $C^\infty(M)$ is a map $D:C^\infty(M)\to C^\infty(M)$ such that
    \begin{enumerate}[label=(\roman*)]
        \item $\mathbb{R}$-linear: $D(\lambda f+\mu g)=\lambda D(f)+\mu D(g)$ for any $\lambda,\mu\in\mathbb{R}$ and $f,g\in C^\infty(M)$,
        \item Leibniz rule: $D(fg)=D(f)g+fD(g)$ for any $f,g\in C^\infty(M)$.
    \end{enumerate}
}

\prop{Smooth Vector Field is Exactly a Derivation on $C^\infty(M)$}{
    Let $M$ be a smooth manifold and $X\in\mathfrak{X}(M)$. Then $X$ can induce a map
    \[
        \begin{aligned}
            X:C^\infty(M)&\longrightarrow C^\infty(M)\\
            f&\longmapsto \left(Xf:p\longmapsto X_p(f)\right)
        \end{aligned}
    \]
    which is a derivation on $C^\infty(M)$.\\
    Conversely, given a derivation $D$ on $C^\infty(M)$, we can define a vector field
    \[
        \begin{aligned}
            X_D:M&\longrightarrow TM\\
            p&\longmapsto X_{D,p}
        \end{aligned}
    \]
    where $X_{D,p}(f)=D(f)(p)$ for any $f\in C^\infty(M)$. Then $X_D$ is a smooth vector field on $M$.
}

\section{Differential}
After defining the tangent space, we can finally define thedifferential for smooth maps.


Let $f:M_1\to M_2$ be a smooth map between smooth manifolds $(M_1,\tau_1,\Sigma_1)$ and $(M_2,\tau_2,\Sigma_2)$. Suppose the mapping is smooth at point $p$. Let $\gamma$ be a curve on $M_1$ passing through point $p$, with its corresponding tangent vector $v_{\gamma}=\left.\frac{d}{\d t}(\cdot \circ \gamma)\right|_0$. By applying $f$ to $\gamma$, we can obatian another curve $f(\gamma)$ on $M_2$ passing through point $f(p)$, with its corresponding tangent vector $v_{f(\gamma)}=\left.\frac{d}{\d t}(\cdot \circ f(\gamma))\right|_0$. 
\lemm{}{
    Choose coordinate charts $(U,\mathbf{x})\in\Sigma_1$ and $(V,\mathbf{y})\in\Sigma_2$ at points $p$ and $f(p)$, respectively. Let the corresponding coordinate functions be $ x^j=p_j\circ\mathbf{x}^{-1}$ and $y^i=p_i\circ\mathbf{y}^{-1}$. Then we can express $v_{f(\gamma)}$ in terms of the natural basis $v_{\eta_1},v_{\eta_2},\cdots,v_{\eta_m}$ of $T_{f(p)}M_2$ as follows: 
    \[
        v_{f(\gamma)}=\sum_{i=1}^m\sum_{j=1}^n\left.\frac{\partial(y^i \circ
        f\circ\mathbf{x})}{\partial
        x^j}\right|_{\mathbf{x}^{-1}(p)}v_\gamma\left( x^j\right)v_{\eta_i}.
        \]
}
\pf{
    \[
        \begin{aligned}
            v_{f(\gamma)}&=\left.\frac{\d(\cdot \circ
        f(\gamma))}{\d t}\right|_{0}\\
            &=\sum_{i=1}^m\left.\frac{\d(y^i \circ
        f(\gamma))}{\d t}\right|_0v_{\eta_i}\\
            &=\sum_{i=1}^m\left.\frac{\d((y^i \circ
        f\circ\mathbf{x})\circ(\mathbf{x}^{-1}\circ\gamma))}{\d t}\right|_0v_{\eta_i}\\
            &=\sum_{i=1}^m\left.\frac{\d(y^i \circ
        f\circ\mathbf{x})}{dx}\right|_{\mathbf{x}^{-1}(p)}\left.\frac{\d(\mathbf{x}^{-1}\circ\gamma)}{\d t}\right|_0v_{\eta_i}\\
            &=\sum_{i=1}^m\sum_{j=1}^n\left.\frac{\partial(y^i \circ
        f\circ\mathbf{x})}{\partial
        x^j}\right|_{\mathbf{x}^{-1}(p)}\left.\frac{\d( x^j\circ\gamma)}{\d t}\right|_0v_{\eta_i}\\
        &=\sum_{i=1}^m\sum_{j=1}^n\left.\frac{\partial(y^i \circ
        f\circ\mathbf{x})}{\partial
        x^j}\right|_{\mathbf{x}^{-1}(p)}v_\gamma\left( x^j\right)v_{\eta_i}.
        \end{aligned}
    \]
}
\cor{}{
    For any curves $\gamma_1$ and $\gamma_2$ passing through point $p$,
    \[
        v_{\gamma_1}=v_{\gamma_2}\implies v_{f(\gamma_1)}=v_{f(\gamma_2)}.
    \]
}

\dfn{Differential of a Smooth Map}{
    The \textbf{differential} of a smooth map $f:M_1\to M_2$ at point $p$ is defined as the following mapping
    \[
    \begin{aligned}
        (\d f)_p: T_pM_1&\longrightarrow T_{f(p)}M_2\\
        v_{\gamma}&\longmapsto v_{f(\gamma)}\\
    \end{aligned}
    \]
    Alternatively, we can utilize the definition based on sheaf theory
    \[
    \begin{aligned}
        (\d f)_p: T_pM_1&\longrightarrow T_{f(p)}M_2\\
        v &\longmapsto v\left(\cdot\circ f\right)\\
    \end{aligned}
    \]
}
Corollary 2.3.1 guarantees that the differential is well-defined. Lemma 2.3.1 implies that the differential is linear. 
\dfn{Tangent Space Functor}{
    \[
        \begin{tikzcd}[ampersand replacement=\&]
            \mathsf{Diff}^\infty_* \&[-25pt] \& [+10pt] \& [-25pt]\mathbb{R}\text{-}\mathsf{Vec}\\[-14pt] 
            (M_1,p) \arrow[dd, "f"{name=L, left}] \& \& \& T_{p}M_1\arrow[dd, "(df)_p"] \\ [-10pt] 
                                            \&  \phantom{.}\arrow[r, "T_{*}", squigarrow]\&\phantom{.}  \&   \\[-10pt] 
            (M_2,f(p)) \& \& \& T_{f(p)}M_2
        \end{tikzcd}
    \] 
}

\dfn{Tangent Map}{
    The \textbf{(global) tangent map} of a smooth map $f:M_1\to M_2$ is defined as the following mapping
    \[
    \begin{aligned}
        \d f: TM_1&\longrightarrow TM_2\\
        (p,v) &\longmapsto \left(f(p),v\left(\cdot\circ f\right)\right)\\
    \end{aligned}
    \]
}

\prop{Tangent Map is Smooth}{
    The tangent map $\d f$ is a smooth map.
}

\dfn{Tangent Bundle Functor}{
    \[
        \begin{tikzcd}[ampersand replacement=\&]
            \mathsf{Diff}^\infty \&[-25pt] \& [+10pt] \& [-25pt]\mathsf{Diff}^\infty\\[-14pt] 
            M_1 \arrow[dd, "f"{name=L, left}] \& \& \& TM_1\arrow[dd, "df"] \\ [-10pt] 
                                            \&  \phantom{.}\arrow[r, "T", squigarrow]\&\phantom{.}  \&   \\[-10pt] 
            M_2 \& \& \& TM_2
        \end{tikzcd}
    \] 
}


\ex{The Tangent Space to an Open Submanifold}{
    Let $M$ be a smooth manifold and $N$ be an open submanifold of $M$. Let $\iota: N \to M$ be the inclusion map. Then for any point $p\in N$, the differential 
    \[
        \begin{aligned}
            \left(\d \iota\right)_p: T_pN&\longrightarrow T_pM\\
            v &\longmapsto v\left(\cdot\circ \iota \right)\\
        \end{aligned}
        \]
    is an isomorphism. 
}

\ex{The Tangent Space to a Finite-Dimensional Vector Space}{
    Let $V$ be a finite-dimensional vector space over $\mathbb{R}$ with its standard smooth structure, and let $p$ be a point in $V$. For any $v\in V$, we can define the directional derivative of a smooth function $f$ at point $p$ in the direction of $v$ as
    \begin{align*}
        \left. D_v\right|_p:C^\infty_p&\longrightarrow \mathbb{R} \\
        [f]_p&\longmapsto \left.\frac{\d f(p+tv)}{\d t}\right|_0 
    \end{align*}
    And we can verify that $\left. D_v\right|_p$ satisfies the Leibniz rule as follows:
    \begin{align*}
        \left. D_v\right|_p([f]_p[g]_p)&=\left.\frac{\d \left(f(p+tv)g(p+tv)\right)}{\d t}\right|_0\\
        &=\left.\frac{\d f(p+tv)}{\d t}\right|_0g(p)+f(p)\left.\frac{\d g(p+tv)}{\d t}\right|_0\\
        &=\left. D_v\right|_p([f]_p)[g]_p+[f]_p\left. D_v\right|_p([g]_p),
    \end{align*} 
    which means that $\left. D_v\right|_p\in T_pM$. Therefore, we can define a mapping
    \begin{align*}
        D_{V,p}:V&\longrightarrow T_pV\\
        v&\longmapsto \left. D_v\right|_p
    \end{align*}
    which is a natural isomorphism between linear spaces $V$ and $T_pM$. The naturality is shown as the following commutative diagram:
    \begin{equation*}
        \begin{tikzcd}[ampersand replacement=\&, row sep=3em]
            V \arrow[d, "D_{V,p}"'] \arrow[r, "f"] \& W \arrow[d, "D_{W,f(p)}"] \\
            T_pV \arrow[r, "(\d f)_p"'] \& T_{f(p)}W
        \end{tikzcd} 
    \end{equation*}
}
The natural basis can also be defined without the use of curves. 

\dfn{Natural Basis}{
    Let $M$ be a smooth manifold with a chart $(U,\mathbf{x})$ and $p\in U$. Then we have the following isomorphisms:
    \[
        T_pM\xleftarrow{(\d \iota)_p}T_pU\xrightarrow{(\d\mathbf{x})_p}T_{\mathbf{x}(p)} \mathbb{R}^n\xleftarrow{D_{\mathbb{R}^n,\mathbf{x}(p)}}\mathbb{R}^n.
    \] 
    Therefore, we can define the \textbf{natural basis} of $T_pM$ under the chart $(U,\mathbf{x})$ as the image of the standard basis of $\mathbb{R}^n$ under the isomorphism 
    $$
     (\d \iota)_p\circ (\d\mathbf{x})_p^{-1}\circ D_{\mathbb{R}^n,\mathbf{x}(p)}.
     $$
    Let $e_i$ be the $i$-th element of the standard basis of $\mathbb{R}^n$, then the $i$-th element of the natural basis is
    \[
        (\d \iota)_p\circ (\d\mathbf{x})_p^{-1}\circ D_{\mathbb{R}^n,\mathbf{x}(p)}(e_i)=\left.\frac{\partial \left(\cdot \circ \mathbf{x}^{-1}\right)}{\partial x^i}\right|_{\mathbf{x}(p)},
    \]
    which coincides with our definition of the natural basis using curves.
}
\pf{
    Let's compute the image of $e_i$ step by step. For any smooth function $f:\mathbb{R}^n\to\mathbb{R}$, we have
    \[
        D_{\mathbb{R}^n,\mathbf{x}(p)}(e_i)(f)=\left.D_{e_i}\right|_{\mathbf{x}(p)}(f)=\left.\frac{\d f(\mathbf{x}(p)+te_i)}{\d t}\right|_0 =\left.\frac{\partial }{\partial x^i}\right|_{\mathbf{x}(p)}(f).
    \]
    Since 
    \[
        (\d\mathbf{x})_p\left(\left.\frac{\partial \left(\cdot \circ \mathbf{x}^{-1}\right)}{\partial x^i}\right|_{\mathbf{x}(p)}\right)(f)=\left.\frac{\partial \left(f \circ \mathbf{x}\circ \mathbf{x}^{-1}\right)}{\partial x^i}\right|_{\mathbf{x}(p)}=\left.\frac{\partial }{\partial x^i}\right|_{\mathbf{x}(p)}(f),
    \] 
    we have
    \[
    (\d\mathbf{x})_p^{-1}\left(\left.\frac{\partial }{\partial x^i}\right|_{\mathbf{x}(p)}\right)=\left.\frac{\partial \left(\cdot \circ \mathbf{x}^{-1}\right)}{\partial x^i}\right|_{\mathbf{x}(p)}.
    \] 
    Note 
    \[
        (\d \iota)_p\left(\left.\frac{\partial \left(\cdot \circ \mathbf{x}^{-1}\right)}{\partial x^i}\right|_{\mathbf{x}(p)}\right)= \left.\frac{\partial \left(\cdot \circ \iota\circ \mathbf{x}^{-1}\right)}{\partial x^i}\right|_{\mathbf{x}(p)}=\left.\frac{\partial \left(\cdot \circ \mathbf{x}^{-1}\right)}{\partial x^i}\right|_{\mathbf{x}(p)}.
    \]
    Therefore, we complete the proof.
}
\prop{}{
    The matrix representation of the tangent map $(df)_p$ in terms of the natural basis is the Jacobian matrix of the function $\mathbf{y}^{-1}\circ f\circ\mathbf{x}$ at $\mathbf{x}^{-1}(p)$, i.e.
    \[
        J_pf:=\begin{pmatrix}
            J_{i,j}
         \end{pmatrix}_{m\times n},\qquad J_{i,j}=\left.\frac{\partial(y^i \circ f\circ\mathbf{x}^{-1})}{\partial x^j}\right|_{\mathbf{x}(p)}.
        \]
}
\pf{
    We choose coordinate charts $(U,\mathbf{x})\in\Sigma_1$ and $(V,\mathbf{y})\in\Sigma_2$ at $p$ and $f(p)$, respectively. The natural bases of $T_pM_1$ and $T_{f(p)}M_2$ are denoted by $v_{\varepsilon_1},v_{\varepsilon_2},\cdots,v_{\varepsilon_n}$ and $v_{\eta_1},v_{\eta_2},\cdots,v_{\eta_m}$, respectively. From Lemma 2.3.1, we have
    \[
        \begin{aligned}
            v_{f(\varepsilon_k)}&=\sum_{i=1}^m\sum_{j=1}^n\left.\frac{\partial(y^i \circ f\circ\mathbf{x})}{\partial x^j}\right|_{\mathbf{x}(p)}v_{\varepsilon_k}\left( x^j\right)v_{\eta_i}\\
            &=\sum_{i=1}^m\sum_{j=1}^n\left.\frac{\partial(y^i \circ f\circ\mathbf{x})}{\partial x^j}\right|_{\mathbf{x}(p)}\delta_{jk}v_{\eta_i}\\
            &=\sum_{i=1}^m\left.\frac{\partial(y^i \circ f\circ\mathbf{x})}{\partial x_k}\right|_{\mathbf{x}(p)}v_{\eta_i}\\
            &=\begin{pmatrix}
            v_{\eta_1}&v_{\eta_2}&\cdots&v_{\eta_m}
            \end{pmatrix}
            \begin{pmatrix}
            \left.\frac{\partial(y^1 \circ f\circ\mathbf{x})}{\partial x_k}\right|_{\mathbf{x}(p)}\\
            \left.\frac{\partial(y^2 \circ f\circ\mathbf{x})}{\partial x_k}\right|_{\mathbf{x}(p)}\\
            \vdots\\
            \left.\frac{\partial(y^m \circ f\circ\mathbf{x})}{\partial x_k}\right|_{\mathbf{x}(p)}
            \end{pmatrix}.
        \end{aligned}
    \]
    Hence 
    \[
        \begin{aligned}
            &\;\quad (df)_p
            \begin{pmatrix}
            v_{\varepsilon_1}&v_{\varepsilon_2}&\cdots&v_{\varepsilon_n}
            \end{pmatrix}\\
            &=
            \begin{pmatrix}
            (df)_pv_{\varepsilon_1}&(df)_pv_{\varepsilon_2}&\cdots&(df)_pv_{\varepsilon_n}
            \end{pmatrix}\\
            &=
            \begin{pmatrix}
            v_{f(\varepsilon_1)}&v_{f(\varepsilon_2)}&\cdots&v_{f(\varepsilon_n)}
            \end{pmatrix}\\
            &=
            \begin{pmatrix}
            v_{\eta_1}&v_{\eta_2}&\cdots&v_{\eta_m}
            \end{pmatrix}
            \renewcommand{\arraystretch}{1.5}
            \left.\begin{pmatrix}
                \frac{\partial(y^1 \circ f\circ\mathbf{x})}{\partial x^1}&\frac{\partial(y^1 \circ f\circ\mathbf{x})}{\partial x^2}&\cdots&\frac{\partial(y^1 \circ f\circ\mathbf{x})}{\partial x^n}\\
                \frac{\partial(y^2 \circ f\circ\mathbf{x})}{\partial x^1}&\frac{\partial(y^2 \circ f\circ\mathbf{x})}{\partial x^2}&\cdots&\frac{\partial(y^2 \circ f\circ\mathbf{x})}{\partial x^n}\\
                \vdots&\vdots&&\vdots\\
                \frac{\partial(y^m \circ f\circ\mathbf{x})}{\partial x^1}&\frac{\partial(y^m \circ f\circ\mathbf{x})}{\partial x^2}&\cdots&\frac{\partial(y^m \circ f\circ\mathbf{x})}{\partial x^n}
            \end{pmatrix}\right|_{\mathbf{x}(p)}
            \renewcommand{\arraystretch}{1}\\
            &=
            \begin{pmatrix}
            v_{\eta_1}&v_{\eta_2}&\cdots&v_{\eta_m}
            \end{pmatrix}
            \begin{pmatrix}
           J_{i,j}
           \end{pmatrix}_{m\times n},
        \end{aligned}
    \]
    where
    \[
        J_{i,j}=\left.\frac{\partial(y^i \circ f\circ\mathbf{x})}{\partial x^j}\right|_{\mathbf{x}(p)}
        \]
}
\cor{Jacobians of Smooth Functions}{
    Let $M$ be a smooth $n$-dimensional manifold and $f:M\to \mathbb{R}$ be a smooth function. Suppose $(U,\mathbf{x})$ is a coordinate chart of $M$ containing point $p$. The differential $(\d f)_p$ has the matrix representation under charts $(U,\mathbf{x})$ and $(\mathbb{R},\mathrm{id}_\mathbb{R})$ as follows:
    \[
        \begin{aligned}
        (\d f)_p
        \begin{pmatrix}
        \left.\frac{\partial }{\partial x^1}\right|_p&\left.\frac{\partial }{\partial x^2}\right|_p&\cdots&\left.\frac{\partial }{\partial x^n}\right|_p
        \end{pmatrix}
        &=
        \begin{pmatrix}
            \left.\frac{\partial}{\partial y}\right|_{f(p)}
        \end{pmatrix}
        \left.\begin{pmatrix}
            \frac{\partial(f\circ\mathbf{x}^{-1})}{\partial
    x^1}&\frac{\partial(f\circ\mathbf{x}^{-1})}{\partial
    x^2}&\cdots&\frac{\partial(f\circ\mathbf{x}^{-1})}{\partial x^n}
        \end{pmatrix}\right|_{\mathbf{x}(p)}
        \end{aligned}
    \]
    Especially, if $f$ is the coordinate function $x^i:M\to \mathbb{R}$, then
    \[
        \begin{aligned}
        \left(\d x^i\right)_p
        \begin{pmatrix}
        \left.\frac{\partial }{\partial x^1}\right|_p&\left.\frac{\partial }{\partial x^2}\right|_p&\cdots&\left.\frac{\partial }{\partial x^n}\right|_p
        \end{pmatrix}
        &=
        \begin{pmatrix}
            \left.\frac{\partial}{\partial y}\right|_{x^i(p)}
        \end{pmatrix}
        \left.\begin{pmatrix}
            \frac{\partial({x}^i\circ\mathbf{x}^{-1})}{\partial
    x^1}&\frac{\partial({x}^i\circ\mathbf{x}^{-1})}{\partial
    x^2}&\cdots&\frac{\partial({x}^i\circ\mathbf{x}^{-1})}{\partial x^n}
        \end{pmatrix}\right|_{\mathbf{x}(p)}\\
        &=\begin{pmatrix}
            \left.\frac{\partial}{\partial y}\right|_{{x}^i(p)}
        \end{pmatrix}
        \begin{pmatrix}
            \delta^i{}_{\hspace{-1pt}1} &\delta^i{}_{\hspace{-1pt}2}&\cdots&\delta^{i}{}_{\hspace{-2pt}n}
        \end{pmatrix}.
        \end{aligned}
    \]
}

\subsection{Cotangent Vector and Cotangent Space}
\subsubsection{Definition}
The most common way to define the cotangent space is to use the dual space of the tangent space.
\dfn{Cotangent Space}{
    The \textbf{cotangent space} at point $p$ of a manifold is the dual space of the tangent space $T_pM$, denoted as $T_p^*M$. Vectors in the cotangent space are called \textbf{cotangent vectors}.
}
If we wish to define the cotangent space directly without the assistance of tangent space, we can start with $C^\infty_p$ and considering a subalgebra of $C^\infty_p$:
\[
    \mathfrak{m}_p=\{f\in C^\infty_p\mid f(p)=0\}.
\]
$\mathfrak{m}_p$ can be considered as the set of functions that vanish at point $p$. We can imagine functions in $\mathfrak{m}_p$ have Taylor expansions as follows:
\[
    f(x)=\sum_{i=1}^n\frac{\partial f}{\partial x^i}(p)x^i+\sum_{i=1}^n\sum_{j=1}^n\frac{\partial^2 f}{\partial x^i\partial x^j}(p)x^ix^j+\cdots.
\] 
Then we define a subalgebra of $\mathfrak{m}_p$:
\[
    \mathfrak{m}^2_p=\left\{\sum_if_ig_i\mid f_i, g_i\in \mathfrak{m}_p\right\},
\]
We can imagine functions in $\mathfrak{m}_p^2$ have Taylor expansions as follows:
\[
    f(x)=\sum_{i=1}^n\sum_{j=1}^n\frac{\partial^2 f}{\partial x^i\partial x^j}(p)x^ix^j+\cdots.
\] 
The cotangent space of the manifold at point $p$ can be defined as $T_p^*M=\mathfrak{m}_p/\mathfrak{m}^2_p$, which only keeps the first order terms in the Taylor expansion.

\prop{Differential as a Cotangent Vector}{
    Let $f\in C^\infty(M)$ and $p\in M$. Then the differential $(\d f)_p$ can be considered as a cotangent vector in $T_p^*M$ by composing it with the natural isomorphism $T_{f(p)}\mathbb{R}\cong \mathbb{R}:\lambda \left.\partial\right|_{f(p)}$
    \[
        T_p M \xrightarrow{(\d f)_p} T_{f(p)}\mathbb{R}\xrightarrow{\left(D_{\mathbb{R},f(p)}\right)^{-1}} \mathbb{R}
    \]

}

\dfn{Cotangent Bundle}{
    The \textbf{cotangent bundle} of a manifold $M$ is the disjoint union of the cotangent spaces of $M$
    \[
        T^*M=\bigsqcup_{p\in M}T_p^*M
    \]
    with the projection map $\pi:T^*M\to M$ defined as $\pi(\omega)=p$ for any $\omega\in T_p^*M$.
}

% \subsubsection{Dual Basis of the Natural Basis}
% Since $T_pM\subset \left(C^\infty_p\right)^*$, we have $T_p^*M\subset \left(C^\infty_p\right)^{**}$. Define the \textbf{natural pairing} $\langle\cdot,\cdot\rangle$, which is a bilinear function:
% \[
%     \begin{aligned}
%     \langle\cdot,\cdot\rangle:T_pM\times C^\infty_p&\longrightarrow\mathbb{R}\\
%     (v_\gamma,f)&\longmapsto\langle v_\gamma,f\rangle:=v_\gamma(f)=\left.\frac{\d(f\circ \gamma)}{\d t}\right|_{0}.
%     \end{aligned}
% \]
% For any given $f\in C^\infty_p$, define
% \[
%     \begin{aligned}
%         \langle\cdot,f\rangle:T_pM&\longrightarrow\mathbb{R}\\
%         v_\gamma&\longmapsto\langle v_\gamma,f\rangle,
%     \end{aligned}
% \]
% we have $\langle\cdot,f\rangle\in T_p^*M$. Conversely, according to the Riesz representation theorem, we can prove that for any $w^*\in T_p^*M$, there exists $f\in C^\infty_p$ such that $w^*=\langle\cdot,f\rangle$. If we consider the linear mapping
% \[
%     \begin{aligned}
%         h:C^\infty_p&\longrightarrow T_p^*M\\
%         f&\longmapsto\langle\cdot,f\rangle,
%     \end{aligned}
% \]
% then we have $h(C^\infty_p)=T_p^*M$.

% Notice that
% \[
%     \langle v_{\varepsilon_i}, x^j\rangle=\left.\frac{\d( x^j\circ\varepsilon_i)}{\d t}\right|_0=\delta^i_{j}\qquad(i=1,2\cdots,n,\;j=1,2\cdots,n).
% \]

% Covector $\langle\cdot,\tilde{x}_1\rangle,\langle\cdot,\tilde{x}_2\rangle,\cdots,\langle\cdot,\tilde{x}_n\rangle$ is the dual basis of the natural basis.

\subsubsection{Dual Basis and Tangent Map}
The dual basis $\langle\cdot,{x}^i\rangle$ and tangent map are intimately related. 
Next we choose a coordinate chart $(U,\mathbf{x})$ of $M$ and consider the tangent map of the coordinate function ${x}^i: M\to \mathbb{R}$:
\[
    \begin{aligned}
        d{x}^i:T_pM&\longrightarrow T_{{x}^i(p)}\mathbb{R}\\
        v&\longmapsto \left(f:\mathbb{R}\to \mathbb{R} \longmapsto\langle v,f\circ {x}^i\rangle\right).
    \end{aligned}
\]
If we endow $\mathbb{R}$ with the standard coordinate chart $(\mathbb{R},\mathrm{id})$, then the matrix representation of $d{x}^i$ under charts $(U,\mathbf{x})$ and $(\mathbb{R},\mathrm{id})$ is
\[
    \begin{aligned}
    (d{x}^i)_p
    \begin{pmatrix}
    v_{\varepsilon_1}&v_{\varepsilon_2}&\cdots&v_{\varepsilon_n}
    \end{pmatrix}
    &=
    \begin{pmatrix}
        \left.\frac{\partial}{\partial y}\right|_{{x}^i(p)}
    \end{pmatrix}
    \left.\begin{pmatrix}
        \frac{\partial({x}^i\circ\mathbf{x})}{\partial
x_1}&\frac{\partial({x}^i\circ\mathbf{x})}{\partial
x_2}&\cdots&\frac{\partial({x}^i\circ\mathbf{x})}{\partial x_n}
    \end{pmatrix}\right|_{\mathbf{x}^{-1}(p)}\\
    &=\begin{pmatrix}
        \left.\frac{\partial}{\partial y}\right|_{{x}^i(p)}
    \end{pmatrix}
    \begin{pmatrix}
        \delta_{i1} &\delta_{i2}&\cdots&\delta_{in}
    \end{pmatrix}.
    \end{aligned}
    \]
Define
\[
    \begin{aligned}
        \mathrm{coor}^{v_{\tau}}:\mathbb{R}&\longrightarrow\mathbb{R}\\
        y&\longmapsto y\delta_{i1},
    \end{aligned}
\]

% \begin{Theorem}{Title}{label}
%     This is the statement of the theorem.
% \end{Theorem}

% \begin{corollary}{Title}{label}
%     This is the statement of the corollary.
% \end{corollary}

% \begin{claim}{Title}{label}
%     This is the statement of the claim.
% \end{claim}

% \begin{Example}{Title}{label}
%     This is an example.
% \end{Example}

% \begin{Definition}{Title}{label}
%     This is a definition.
% \end{Definition}
\end{document}