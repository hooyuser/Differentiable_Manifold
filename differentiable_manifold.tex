\documentclass{report}

\input{preamble}


\begin{document}

% Your document content here
\chapter{Basic Concepts of Manifolds}
\section{Topological Manifold}
\subsection{Definition of Topological Manifold}
\dfn{Topological Manifold}{
    An $n$-dimensional \emph{topological manifold} is a Hausdorff topological space that is locally homeomorphic to $n$-dimensional Euclidean space.
}
``Locally homeomorphic to $n$-dimensional Euclidean space" means that for any point $p$ in the topological space $M$, there exists a neighborhood $N(p)$ of $p$ such that $N(p)$ is homeomorphic to the Euclidean space $\mathbb{R}^n$.

\subsection{Coordinate Chart}

\dfn{Coordinate Chart}{
    If $U_\alpha$ is an open set in $\mathbb{R}^n$,
    $$
        \mathbf{x}_\alpha:U_\alpha\longrightarrow \mathbf{x}_\alpha (U_\alpha)\subseteq M
    $$
    is a homeomorphism and $\mathbf{x}_\alpha (U_\alpha)$ contains some point $p\in M$, then the pair $(U_\alpha,\mathbf{x}_\alpha)$ is called a \emph{coordinate chart} or \emph{(local) coordinate system} of the topological manifold $M$ at point $p$.
}

Given a coordinate chart, $\mathbf{x}^{-1}_{\alpha}$ is often called the local coordinate mapping at point $p$. The image of point $p$ under $\mathbf{x}^{-1}_{\alpha}$, denoted as $\mathbf{x}^{-1}_\alpha(p)$, is called the local coordinate of point $p$. Coordinate charts can transform the local structure of a topological manifold into the familiar coordinates in $\mathbb{R}^n$, so they are sometimes also called parameterizations.

At any point $p$ in the topological manifold $M$, there exists at least one coordinate chart. In fact, let $\eta$ be a homeomorphism from $N(p)$ to $\mathbb{R}^n$. By definition of a neighborhood, there exists an open set $W\subseteq M$ such that $p\in W\subseteq N(p)$. Therefore, if we restrict $\eta$ to $W$ and let $\mathbf{x}=(\left.\eta\right|_{W})^{-1}$, then
$$
    \mathbf{x}:U\longrightarrow \mathbf{x}(U)
$$
is also a homeomorphism. Hence, $(U,\mathbf{x})$ is a coordinate chart at point $p$.



% \begin{Theorem}{Title}{label}
%     This is the statement of the theorem.
% \end{Theorem}

% \begin{corollary}{Title}{label}
%     This is the statement of the corollary.
% \end{corollary}

% \begin{claim}{Title}{label}
%     This is the statement of the claim.
% \end{claim}

% \begin{Example}{Title}{label}
%     This is an example.
% \end{Example}

% \begin{Definition}{Title}{label}
%     This is a definition.
% \end{Definition}
\end{document}