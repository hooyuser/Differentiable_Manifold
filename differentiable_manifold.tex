\documentclass{report}

\input{preamble}


\begin{document}

% Your document content here
\chapter{Basic Concepts of Manifolds}
\section{Topological Manifold}
\subsection{Definition of Topological Manifold}

\begin{Definition}{topological manifold}
    An $n$-dimensional \emph{topological manifold} is a Hausdorff topological space which is locally homeomorphic to an $n$-dimensional Euclidean space.
\end{Definition}

The statement “locally locally homeomorphic to an $n$-dimensional Euclidean space” means that given any point $p$ in the topological space $M$, there exists a neighborhood of $p$, $N(p)$, such that $N(p)$ is locally homeomorphic to the Euclidean space $\mathbb{R}^n$.

\subsection{Coordinate Chart}

% \begin{Definition}[coordinate chart]
% If $U_\alpha$ is an open set in $\mathbb{R}^n$,
% $$
% \mathbf{x}_\alpha:U_\alpha\to \mathbf{x}_\alpha (U_\alpha)\subset M
% $$
% is a homeomorphism and $\mathbf{x}_\alpha (U_\alpha)$ contains $p$, then the two-tuple $$(U_\alpha,\mathbf{x}_\alpha)$$ is said to be a \emph{coordinate chart} of $M$ at the point $p$.
% \end{Definition}

% Given a coordinate chart, $\mathbf{x}^{-1}_{\alpha}$ is often called the local coordinate mappingat thr point $p$. The image of the point $p$ under $\mathbf{x}^{-1}_{\alpha}$, namely $$\mathbf{x}^{-1}_\alpha(p)$$ is called the local coordinate of $p$. Coordinate charts can transform the local part of a topological manifold into the coordinates in the familiar $\mathbb{R}^n$ , so they are sometimes called parameterization.

% Topological manifold $M$ has at least one coordinate card at any point $p$. In fact, let $$\eta$$ be a homeomorphism from $$N(p)$$ to $$\mathbb{R}^n$$. According to the definition of neighborhood, there is an open set $$W$$ in $M$ such that $$p\in W\subset N(p)$$. Therefore, if $$\eta$$ is limited to $$W$$ and $$\mathbf{x}=(\left.\eta\right|_{W})^{-1}$$, then 
% $$
% \mathbf{x}:U\to \mathbf{x}(U)
% $$ 
% is also a homeomorphism. Hence, $(U,\mathbf{x})$ is a coordinate chart at the point $p$. 



\begin{Theorem}{Title}{label}
    This is the statement of the theorem.
\end{Theorem}

\begin{corollary}{Title}{label}
    This is the statement of the corollary.
\end{corollary}

\begin{claim}{Title}{label}
    This is the statement of the claim.
\end{claim}

\begin{Example}{Title}{label}
    This is an example.
\end{Example}

\begin{Definition}{Title}{label}
    This is a definition.
\end{Definition}
\end{document}